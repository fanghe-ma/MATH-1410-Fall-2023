\chapter{Week 7}

\section{Gradient}

\subsection{The level set}

\begin{framed}
   For functions of the form \[
     f: \mathbb{R}^m \rightarrow R
   \] 

   We can consider its level sets, which are subsets of the domain given by the implicit equations
   \[
     f( \underline{x}) = c
   \] for some constant $c$

   We denote level sets with
   \[
      f^{-1} (c) = { \underline{x} : f( \underline{x}) = c}
   \] 

   \textbf{Note} this does not imply $f$ is invertible
\end{framed}

\subsection{Definition of the gradient}

For scalar-value functions, there is a more primitive approach to the derivative

\begin{framed}
   For $ f: \mathbb{R}^n \rightarrow \mathbb{R}$, the \textbf{gradient} is the \textbf{vectorized derivative}

   \[
     \nabla f = \begin{bmatrix} 
        \frac{\partial f}{\partial x_1} \\
        \frac{\partial f}{\partial x_2} \\
        \vdots \\
        \frac{\partial f}{\partial x_n}
     \end{bmatrix}
   \] 
   Note that: 
   \begin{itemize}
      \item $\nabla f$ is a \textbf{vector} at $\underline{a}$
      \item $ \left[ D f \right]_{ \underline{a}} $ is a \textbf{linear transformation} acting on vectors at $\underline{a}$
   \end{itemize}
\end{framed}

\subsection{Gradients and level sets}

\begin{framed}
   \textbf{Lemma}: At each point, the gradient is \textbf{orthogonal} to that point's level set and is oriented in the direction of \textbf{maximally increasing value}
\end{framed}

\textbf{Proof}

For a point $ \underline{a}$ where $ f( \underline{a}) = c$, let $ \underline{u}$ be a unit vector at $ \underline{a}$ \\

Let the rate of change of $ f$ along $ \underline{u}$ be denoted by $ D_{ \underline{u}}f$. \\

Recall that the rate of change of outputs can be given by \[
   D_{ \underline{u}}f = \left[ D f \right]_{ \underline{a}}  \underline{u}  = \nabla f \cdot \underline{u}
\] 

Since $ \left[ D f \right]_{} \underline{u}  $ is equivalent to $ \nabla f \cdot \underline{u}$  and $ \nabla f \cdot u$ is the length of projection of $ \nabla f$ onto $ \underline{u}$, we can conclude that 

\begin{itemize}
   \item when $ \underline{u}$ is tangent to $ f^{-1}(c)$, $f$ is not changing, hence \[
         \nabla f \bot f^{-1} (c)
   \] 
   \item when $ \underline{ u}$ points along $\nabla f$, the rate of change is maximized, hence gradient points in the direction of maximal increasing value
\end{itemize}

\subsection{Directional Derivatives}

There is only one derivative, but another set of terminology exists

\begin{framed}
   For a real value function, to compute the derivative along a direction given by the unit vector $\underline{u}$, the \textbf{directional derivative} is defined as \[
      D_{ \underline{u}}f = \nabla f \cdot \underline{u} = \left[ D f \right]_{} \underline{u}
   \]  
\end{framed}

\section{Tangent Spaces}

\subsection{The tangent space of an implicit surface}

For an implicit surface in $3d$ given by $F(x, y, z) = 0$, we can find the tangent plane to a point $ \underline{x}_0$ by using the gradient
\[
  \nabla F \cdot ( \underline{x} - \underline{x}_0) = 0
\] 

\subsection{The tangent space of an parameterized curve}

For a parameterized curve
\[
  f(t) =  \begin{pmatrix} 
    x(t) \\ y(t) \\ z(t)  
  \end{pmatrix}
\] 

The velocity is
\[
  \left[ D f \right]_{}  = \begin{pmatrix} 
    \frac{dx}{dt} \\ \frac{dy}{dt} \\ \frac{dz}{dt}  
  \end{pmatrix}
\] 

The tangent line at $t_0$ is parameterized by another variable, $s$. A point on the tangent line can be expressed as 
\[
  \underline{u}(s) = f(t_0) + \left[ D f \right]_{t_0} s
\] 

\subsection{The tangent space of a parameterized surface}

Generalizing to a case in $3d$, for a parameterized surface of the form
\[
  f: \mathbb{R}^2 \rightarrow \mathbb{R}^3
\] 

Given a parameterization
\[
  \underline{x} = f( \underline{t})  
\]  where

\[
  \underline{x} = \begin{pmatrix} x \\ y\\ z \end{pmatrix} , \underline{t} = \begin{pmatrix} t_1 \\t_2 \end{pmatrix} 
\] 

The tangent plane at  $f(t_0)$ is given by the same formula 
\[
  \underline{x} = f( \underline{t}_0) + \left[ D f \right]_{}  \underline{s}
\]
where there are two parameters $s_1$ and $s_2$ in $ \underline{s}$
\[
  s = \begin{pmatrix} s_1 \\ s_2 \end{pmatrix} 
\] 

The two columns in $\left[ Df \right]_{} $ form the basis vectors of the tangent plane, a point in which is specified by two parameters $s_1$ and $s_2$

\textbf{Parametric to implicit!!}: to find the implicit equation of the plane, recall that a plane can be implicitly defined using the normal to the plane, which can be found by taking the cross product of the two columns in $\left[ D f \right]_{} $

\subsection{Images, Kernels, and tangent spaces in higher dimensions}
Bonus content, will fill in later

\section{Linearization}

Finding tangent spaces is essentially a linearization of a function about a certain point, which provides a locally valid approximation

\subsection{Differentials vs derivatives}

Differentials and derivatives provide two different notations for the same idea \\

For a function
\[
  f( \underline{x}) =f(x_1, x_2, \hdots x_n)
\] 

\textbf{Differentials}

$df$ is a linear combination of differentials
\[
   df = \sum_{i=1}^{n} \frac{\partial f}{\partial x_i} dx_i
\] 

\textbf{Derivatives}
\[
\left[ D f \right]_{}  = \frac{\partial f}{\partial \underline{x}} = \begin{bmatrix} 
   \frac{\partial f}{\partial x_1} & \frac{\partial f}{\partial x_2} & \hdots & \frac{\partial f}{\partial x_n}  
\end{bmatrix}
\] 

$ \left[ D f \right]_{} $ is a linear transformation that acts on vectors of rates of change of inputs \\

if $ d \underline{x}$ is a vector of differentials $ x_i$, then the rate of change of output is
\[
   \frac{\partial f}{\partial \underline{x}} d \underline{x} = df
\] 

\subsection{Relative rates of change and percentage errors}

\begin{framed}
   For a quantity $u$, the relative rate of change of u is given by
   \[
     d(ln u) = \frac{du}{u}
   \] 
\end{framed}

The rate of change of output is a linear combination of rate of change of inputs  \\

\textbf{Given}: 
\[
  u( \underline{x)} = u (x_1, x_2, \hdots , x_n)
\] 

Assume that $u$ is of the form
\[
   u = K \Pi_{i=1}^{n} x_i^{c_i}
\] 

Then
\begin{align*}
   \frac{du}{u} &= d (ln u) \\
                &= d(ln k + \sum_{i=1}^{n} ln(x_i^{c_i})) \\
                &= \sum_{i=1}^{n} c_i d(ln(x_i)) \\
                &= \sum_{i=1}^{n} c_i \frac{dx_i}{x_i}\\
\end{align*}

\section{Taylor Series}

\subsection{Single variable case}
\begin{framed}
   Recall that for a function
   \[
     f: \mathbb{R} \rightarrow \mathbb{R}
   \] 

   The Taylor series is
   \begin{align*}
      f(x) &= \sum_{i=0}^{\infty} \frac{1}{i!} \left. \frac{d^i f}{dx^i} \right|_{0} x^i  \\
      f(x) &= \sum_{i=0}^{\infty} \frac{1}{i!} \left. D^i f \right|_{0} x^i  \\
      f(a + h) &= \sum_{i=0}^{\infty} \frac{1}{i!} \left. D^i f \right|_{a} h^i  \\
      f(x) &= \sum_{i=0}^{\infty} \frac{1}{i!} \left. D^i f \right|_{a} (x-a)^i 
   \end{align*}
\end{framed}

\subsection{Multivariable case}

\begin{framed}
   For a function 
   \[
     f : \mathbb{R}^n \rightarrow \mathbb{R}
   \] 

   The Taylor Series is given by
   \[
      f( \underline{x}) = \sum_{I} \frac{1}{I!} \left. D^I f \right|_{ \underline{0}} \underline{x}^I
   \]  

   \[
      f (\underline{x}) = \sum_{I}  \frac{1}{I!} \left. D^I f  \right|_{ \underline{a} } ( \underline{x} - \underline{a})^I
   \] 
   \[
      f (\underline{a} + \underline{h}) = \sum_{I}  \frac{1}{I!} \left. D^I f  \right|_{ \underline{a} } ( \underline{h})^I
   \] 
\end{framed}

\textbf{The multi-index}

  Given $n$ variables
  \[
     \underline{x} = (x_1, x_2, x_3, \hdots x_n)
  \] 

  The multi-index is $n$ ordered indices
  \[
    I = (i_1, i_2, i_3, \hdots, i_n)
  \] 

\textbf{The degree of a multi-index}
  Where the degree is given by
  \[
    |I| = i_1+ i_2 + \hdots + i_n
  \] 

\textbf{Taking a vector to the power of the multi-index}: given $ \underline{x}$ and $I$, 
\[
  \underline{x}^I = x_1^{i_1}x_2^{i_2} \hdots x_{n}^{i_n}
\]


\textbf{Taking the factorial of a multi-index}
For a multi-index $ I$,
\[
  I = (i_1, i_2, \hdots, i_n)
\] 

The factorial is defined as
\[
   I! = i_1!  i_2 ! \hdots i_{n}!
\] 

\textbf{Multi-index and the derivative}

Given \[
  f = f( \underline{x)} = f(x_1, x_2, \hdots , x_n)
\] 

We define \[
   D^I f = \frac{\partial^{i_1} }{\partial x_1^{i_1}} \frac{\partial^{i_2} }{\partial x_2^{i_2}} \hdots \frac{\partial^{i_n} }{\partial x_n^{i_n}} f
\] 

\section{Computing Taylor Series}

\subsection{Commonly used Taylor Series}

\begin{framed}

   \iffalse
 Exponentials
 \renewcommand{\arraystretch}{1.5}
 \[
    \begin{array}{l  c  l l}
      e^x 
      & \text{all } x
      & \sum\limits_{n=0}^{\infty} \frac{x^n}{n!} 
      & = 1 + x + \frac{x^2}{2!} + \frac{x^3}{3!} + \frac{x^4}{4!} + \frac{x^5}{5!} + \frac{x^6}{6!} + \hdots\\
      a^x  = e^{x ln (a)}
      & \text{all } x
      & \sum\limits_{n=0}^{\infty} \frac{(x ln(a))^n}{n!} 
      & = 1 + x ln(a) + \frac{(xln(a))^2}{2!} + \frac{(x ln(a))^3}{3!} + \hdots\\
   \end{array}
 \] 

 Logarithms
 \[
    \begin{array}{l c l l}
      ln(1-x)
      & \text{for } \left| x \right| < 1
      & \sum\limits_{n=1}^{\infty} \frac{1}{n} -x^n
      & = -x - \frac{x^2}{2} -\frac{x^3}{3} - \frac{x^4}{4} - \frac{x^5}{5} - \frac{x^6}{6} - \hdots\\ 
      ln(x)
      & \text{for } \left| x \right| < 1
      & \sum\limits_{n=1}^{\infty} \frac{(-1)^{n-1}}{n}(x-1)^n
      & = (x-1) - \frac{(x-1)^2}{2} + \frac{(x-1)^3}{3} - \frac{(x-1)^4}{4} + \hdots\\
      ln(x + 1)
      & \text{for } \left| x \right| < 1
      & \sum\limits_{n=1}^{\infty} \frac{(-1)^{n-1}}{n}  x^n
      & = x - \frac{x^2}{2} + \frac{x^3}{3} - \frac{x^4}{4} + \frac{x^5}{5} - \frac{x^6}{6}\\
   \end{array}
 \] 

 Geometric series
 \[
    \begin{array}{l  c  l l}
      \frac{1}{x}
      & \text{for } 0 < x < 2 
      & \sum\limits_{n=1}^{\infty} (-1)^n (x-1)^n
      & = 1 - (x - 1) + (x-1)^2 - (x-1)^3 + (x-1)^4\\
      \frac{1}{1+x}
      & \text{for } \left| x \right| < 1
      & \sum\limits_{n=0}^{\infty} (-1)^n (x)^n
      & 1 -x + x^2 - x^3 + x^4 -x^5 + x^6 \\
      \frac{1}{1-x}
      & \text{for } \left| x \right| < 1
      & \sum\limits_{n=0}^{\infty} (x)^n
      &= 1 + x + x^2 + x^3 + x^4 + x^5 + x^6 + \hdots\\
      \frac{1}{1-x}
      & \text{for } \left| x \right| < 1
      & \sum\limits_{n=0}^{\infty} (x)^n
      &= 1 + x + x^2 + x^3 + x^4 + x^5 + x^6 + \hdots\\
      \frac{1}{1+x^2}
      & \text{for } \left| x \right| < 1
      & \sum\limits_{n=0}^{\infty} (-1)^n (x)^{2n}
      &= 1 -x^2 + x^4 - x^6 + x^8  + \hdots\\
      \frac{1}{1-x^2}
      & \text{for } \left| x \right| < 1
      & \sum\limits_{n=0}^{\infty} (x)^{2n}
      &= 1 +x^2 + x^4 + x^6 + x^8  + \hdots\\
      \frac{1}{(1+x)^2}
      & \text{for } \left| x \right| < 1
      & \sum\limits_{n=1}^{\infty} (-1)^{n-1} nx^{n-1}
      &= 1 -2x + 3x^2 -4x^3 + 5x^4 - 6x^5 + 7x^6  + \hdots\\
      \frac{1}{(1-x)^2}
      & \text{for } \left| x \right| < 1
      & \sum\limits_{n=1}^{\infty}  nx^{n-1}
      &= 1 +2x + 3x^2 +4x^3 + 5x^4 + 6x^5 + 7x^6  + \hdots\\
      \sqrt{1 + x}
      & \text{for } -1 < x \leq 1
      & \sum\limits_{n=0}^{\infty} \frac{(-1)^n (2n)!}{4^n (n!)^2 (1-2n)}  x^n
      &=  1 + \frac{1}{2}x - \frac{1}{8}x^2 + \frac{1}{16}x^3 - \frac{5}{128}x^4 + \frac{7}{256}x^5 + \hdots\\
      \sqrt{1 - x}
      & \text{for } -1 < x \leq 1
      & \sum\limits_{n=0}^{\infty} \frac{ (2n)!}{4^n (n!)^2 (1-2n)}  x^n
      &=  1 - \frac{1}{2}x - \frac{1}{8}x^2 - \frac{1}{16}x^3 - \frac{5}{128}x^4 - \frac{7}{256}x^5 - \hdots\\
      \sqrt{1 + x^2}
      & \text{for } -1 < x \leq 1
      & \sum\limits_{n=0}^{\infty} \frac{ (-1)^n (2n)!}{4^n (n!)^2 (1-2n)}  x^{2n}
      &=  1 + \frac{1}{2}x^2 - \frac{1}{8}x^4 + \frac{1}{16}x^6 - \frac{5}{128}x^8 + \frac{7}{256}x^{10} - \hdots\\
      \sqrt{1 - x^2}
      & \text{for } -1 < x \leq 1
      & \sum\limits_{n=0}^{\infty} \frac{ (2n)!}{4^n (n!)^2 (1-2n)}  x^{2n}
      &=  1 - \frac{1}{2}x^2 - \frac{1}{8}x^4 - \frac{1}{16}x^6 - \frac{5}{128}x^8 - \frac{7}{256}x^{10} - \hdots\\
   \end{array}
 \] 

 Binomial Expansion

 Trigonometric functions

 Inverse trigonometric functions

 Hyperbolic functions

 Inverse Hyperbolic functions
\fi

For all $x$
\[
   \begin{array}{l r l}
   e^x &= 1 + x + \frac{x^2}{2!} + \frac{x^2}{3!} + \hdots &= \sum\limits_{n=0}^{\infty}  \frac{x^n}{n!} \\
   sin\ x &= x - \frac{x^3}{3!} + \frac{x^5}{5!} - \frac{x^7}{7!} + \hdots &= \sum\limits_{n=0}^{\infty} \frac{(-1)^n x^{2n+1}}{(2n+1)! } \\
   cos\ x &= 1 - \frac{x^2}{2!} + \frac{x^4}{4!} - \frac{x^6}{6!} + \hdots &= \sum\limits_{n=0}^{\infty}  \frac{(-1)^n x^{2n}}{(2n)!} \\
   sinh\ x &= x + \frac{x^3}{3!} + \frac{x^5}{5!} + \frac{x^7}{7!} + \hdots &= \sum\limits_{n=0}^{\infty} \frac{x^{2n+1}}{(2n+1)! } \\
   cosh\ x &= 1 + \frac{x^2}{2!} + \frac{x^4}{4!} + \frac{x^6}{6!} + \hdots &= \sum\limits_{n=0}^{\infty}  \frac{ x^{2n}}{(2n)!}
\end{array}
\] 

For $x$ in restricted interval

\[
\begin{array}{l r l l}
   \frac{1}{1-x} &= 1 + x + x^2 + x^3 + \hdots &= \sum\limits_{n=0}^{\infty}  x^n & \text{ for $x \in (-1, 1)$} \\
   \frac{1}{(1-x)^2} &= 1 + 2x + 3x^2 + 4x^3 + \hdots &= \sum\limits_{n=1}^{\infty}  nx^{n-1} & \text{ for $x \in (-1, 1)$} \\
   ln(1+x) &= x - \frac{x^2}{2} + \frac{x^3}{3} - \frac{x^4}{4} + \hdots &= \sum\limits_{n=0}^{\infty} \frac{(-1)^n x^{n+1}}{n+1} & \text{ for $x \in (-1, 1]$} \\
   arctan\ x &= x - \frac{x^3}{3} + \frac{x^5}{5} - \frac{x^7}{7} + \hdots &= \sum\limits_{n=0}^{\infty} \frac{(-1)^n x^{2n+1}}{2n+1} & \text{ for $x \in [-1, 1]$} \\
   (1+x)^k &= 1 + kx + \frac{k(k-1)}{2!}x^2 + \frac{k(k-1)(k-2)(k-3)}{3!} + \hdots & & \text{ for $x \in (-1, 1)$} \\
  
\end{array}
\] 



\end{framed}

\subsection{Computing the Taylor Expansion of a function with 2 inputs and 1 output}

\begin{framed}
   For a planar function about the origin, 
   \begin{align*}
      f(x, y) &= f(0,0)  \\
              &+ \left. \frac{\partial f}{\partial x} \right|_{ \underline{0}} x + \left. \frac{\partial f}{\partial y} \right|_{ \underline{0}} y \\
              &+ \frac{1}{2} \left. \frac{\partial^2 f}{\partial x^2} \right|_{ \underline{0}} x^2 
                 + \left. \frac{\partial^2 f}{\partial x \partial y} \right|_{ \underline{0} } xy + \frac{1}{2} \left. \frac{\partial^2f }{\partial y^2} \right|_{ \underline{0}} y^2 \\
              &+ \frac{1}{6} \left. \frac{\partial^3 f}{\partial x^3} \right|_{ \underline{0}} x^3 + \frac{1}{2} \left. \frac{\partial^3 f}{\partial x^2 \partial y} \right|_{ \underline{0}} x^2 y + \frac{1}{2} \left. \frac{\partial^3 f}{\partial x \partial y^2} \right|_{ \underline{0}} xy^2 + \frac{1}{6} \left. \frac{\partial^3 f}{\partial y^3} \right|_{ \underline{0}} y^3
   \end{align*}
\end{framed}

\subsection{Taylor Series and the Chain Rule}
\begin{framed}
   The composition of Taylor Series is the Taylor Series of the composition
\end{framed}

\subsection{Using Taylor Series for local analysis}

Taylor Series can be used to analyze functions locally \\

For example, given a system
\[
   sin(xy) = e^{x^3} - cos(x^2y)
\] 

We can find solutions locally near $(0.0)$ by taking the Taylor series of both sides
\[
  xy - O((xy)^3) = \left( 1+ x^3 + O(X^6) - (1- O(x^2y)^2) \right)
\] 

Simplifying, we get
\[
  xy - x^3 = 0
\] 

\subsection{Taylor Series up to the second order}

\begin{framed}
   \[
     f ( \underline{a} + \underline{h}) = f ( \underline{a}) + \frac{1}{1!} \left[ D f \right]_{ \underline{a}} \underline{h} + \frac{1}{2!} h^T \left[ D^2 f \right]_{ \underline{a}} \underline{h} + O( \lVert h \rVert^3)
   \] 
\end{framed}

\subsection{The Hessian}
\begin{framed}
   \textbf{Definition}: 
   \[
   \left[ D^2 f \right]_{ \underline{a} i, j} = \frac{\partial^2 f}{\partial x_i \partial x_j}
   \] 

   Hence, 

   \[
     \left[ D^2 f \right]_{}  = \begin{bmatrix} 
        \frac{\partial^2 f}{\partial x_1^2} & \hdots & \frac{\partial^2 f}{\partial x_n \partial x_1}   \\
        \vdots & \frac{\partial^2 f}{\partial x_j^2} & \vdots   \\
        \frac{\partial^2 f}{\partial x_1 \partial x_n} & \hdots & \frac{\partial^2 f}{\partial x_n^2 }   \\
     \end{bmatrix}
   \] 
\end{framed}









