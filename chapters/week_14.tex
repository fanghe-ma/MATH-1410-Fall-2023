\chapter{Week 14}

\section{Integrating 2-forms}

\subsection{Computing the integral}
\begin{framed}
A 2-form is a weighted area form.
\[
   \int_{D}^{} f(x, y)  dx \wedge dy = \iint_{D} f(x, y) dA 
\] 
\end{framed}

Recall Green's Theorem

\[
  \int_{\partial D}^{} Pdx + Q dy = \iint \frac{\partial Q}{\partial x} - \frac{\partial P}{\partial y} dx dy 
\] 

If we define
\[
  \alpha = Pdx + Q dy 
\] 
then
\[
  d\alpha = \frac{\partial P}{\partial y} dy\wedge dx + \frac{\partial Q}{\partial x} dx \wedge dy 
\] 

More generally,

\[
  \int_{\partial D}^{}  \alpha = \int_{D}^{} d \alpha  
\] 

To integrate a 2-form field over a parameterized surface defined by $G(s, t)$, a pair of basis vectors in the input is transformed by  $\left[ D G \right]_{(s, t)} $ to a pair of output vectors \\

At each point on the surface, there is a pair of tangent vectors $ \frac{\partial G}{\partial s} $ and $\frac{\partial G}{\partial t}$, the columns of $\left[ D G \right]_{(s, t)} $

\begin{framed}
Given a 2-form field $\beta$, we define the integral
\[
   \int_{S}^{} \beta = \iint_R \beta_{G(s, t)}  \left[ D G \right]_{(s, t)}   ds dt
\] 
\end{framed}

\subsection{Computing flux}

2-forms fields can be thought of as flux forms in 3-D \\

For a 3-D vector field
\[
  \vec{F} = F_x \underline{i} + F_y \underline{j} + F_z \underline{k}
\] 

There is an associated flux 2-form
\[
   \Phi_{\vec{F}} = F_x dy \wedge dz + F_y dz \wedge dx + F_z  dx \wedge dy
\] 

The flux 2-form tells you how much "stuff" is getting pushed through an infinitesimal oriented area  \\

The integral of a flux 2-form gives the net flux, or how much "stuff" is getting pushed through an oriented surface \\

\begin{framed}
   The flux of $\vec{F}$ across $S$ is
   \[
      \int_{S}^{} \vec{F} \cdot \underline{n} d \sigma = \int_{S}^{} \Phi_{\vec{F}} 
   \] 
\end{framed}

\section{Gauss' Theorem}

Recall Green's Theorem \\

\textbf{Flux form}

Integrating the flux 1-form over some domain $D \in \mathbb{R}^2$ with oriented boundary $\partial D$ is the same as integrating the \textbf{2-D} divergence with respect to the area element
\[
  \int_{\partial D}^{}  F_x dy - F_y dx = \iint_D \nabla \cdot \vec{F} dA 
\] 

OR

\[
   \int_{\partial D}^{} \Phi_{\vec{F}} = \int_{D}^{} d \Phi_{\vec{F}}  
\] 

In 3-D, this is Gauss' Theorem

\subsection{The Theorem}
\begin{framed}
   \textbf{Theorem}: Let $D \in \mathbb{R}^3$ be a bounded domain with boundary $\partial D$, oriented by outward-pointing normal $ \underline{n}$. Then, for a smooth vector field $\vec{F}$ on D, 
   \[
      \iint_{\partial D}^{}  \vec{F} \cdot \underline{n} d\sigma = \iiint_{D} \nabla \cdot \vec{F} dV 
   \] 
   OR
   \[
      \int_{\partial D}^{}  \Phi_{\vec{F}} = \int{D} d \Phi_{\vec{F}} 
   \] 
\end{framed}

\section{Stoke's Theorem}

Gauss' Theorem is a 3-D generalization of the \textbf{flux form of Green's Theorem}\\

Now we consider the circulation form of Green's Theorem

For a vector field
\[
  \vec{F} = F_x \underline{i} + F_y \underline{j}
\] 

and $D \in \mathbb{R}^2$ a domain with oriented boundary $\partial D$

\[
  \int_{\partial D}^{} F_x dx + F_y dy = \iint_D \left( \frac{\partial F_y}{\partial x} - \frac{\partial F_x}{\partial y} \right) dA = \iint_D \left( \nabla \times \vec{F} \right) \cdot \underline{k} dA
\] 
Or

\[
   \int_{\partial D}^{}  \alpha_{\vec{F}} = \int_{D}^{} d \alpha_{\vec{F}}  
\] 

\begin{framed}
   \textbf{Theorem}: Let $D \in \mathbb{R}^3$ be a bounded surface with boundary $\partial D$ oriented by a field of unit vectors $ \underline{n} $ normal to $D$, then, for a smooth vector field $ \vec{F}$ on $D$, 
   \[
   \int_{\partial D}^{} \vec{F} \cdot d l = \iint_{D} \nabla \times \vec{F} \cdot \underline{n} d \sigma  
   \] 

   Or

   The circulation of a field around the boundary is equal to the net flux of the curl through the surface 

   Or
   \[
      \int_{\partial D}^{ } \alpha_{\vec{F}} = \int_{D}^{} d \alpha_{\vec{F}} 
   \] 

\end{framed}

\subsection{Independence of surface}

For some fixed $\gamma = \partial D$, the choice of $D$ does not matter as long as orientation remains the same


\section{Which theorem to use?}
