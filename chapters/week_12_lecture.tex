\chapter{Week 12 lectures - Fields}

\section{Fields}

Recall from the whole semester that we have seen

\textbf{scalar fields}, which have the form
\[
  f: \mathbb{R}^n \rightarrow \mathbb{R}
\] 
\textbf{vector fields}, an example of which is the \textbf{gradient}
\[
  \nabla f
\] 

\textbf{ \textit{smiley} fields}, where every point is a \textit{smiley}
\begin{itemize}
   \item smiley can be anything
   \item it can be a matrix, such as in
      \[
        F: \mathbb{R}^n \rightarrow \mathbb{R}^m
      \]
   \item the derivative $\left[ D F \right]_{}$ is a matrix field
\end{itemize}

Moving on to think about integrals, given a scalar field $f: \mathbb{R}^n \rightarrow \mathbb{R}$, we can compute the integral over a domain, or we can compute a path integral

The path can be parameterized by $\gamma(t)$
\[
   \gamma: [a,b] \rightarrow \mathbb{R}^n
\] 

The path integral can be computed by integrating the arc length element, weighted by the scalar field
\begin{align*}
   &\int_{\gamma} f\ dl \\
   &= \int_{\gamma} f\ \left| \gamma ' \right|  dt
\end{align*}

For example, given $r(t) = \begin{pmatrix} t \\ t^2 -1 \end{pmatrix} $ and a scalar field $f = x^2 + y^2 $ for  $0 \leq t \leq 2$, we can compute that
 \[
  dl = \left| \begin{pmatrix} 1 \\ 2t \end{pmatrix}  \right| dt = \sqrt{1 + 4t^2} dt
\] 

The path integral is 
\[
   \int_{\gamma}f\ dl &= \int_{t=0}^{2} \left( t^2 + \left( t^2 -1 \right)^2 \right) \sqrt{1 + 4t^2} dt
\] 

\textbf{Note}: the path integral $\int_{\gamma} f\ dl$ is independent of the parameterization of $\gamma$
 \begin{itemize}
    \item this can be proved via the Change of Variables Theorem
    \item this makes sense intuitively!
\end{itemize}

\section{Something new!! - the 1-form}

Recall that for a scalar field $f$, we have the differential $df$ 
\[
  df = \frac{\partial f}{\partial x_1} dx_1 + \frac{\partial f}{\partial x_2} dx_2 + \hdots + \frac{\partial f}{\partial x_n} dx_n
\] 

The question we must ask ourselves is
\[
   \text{What is $dx$?}
\] 

It is a \textbf{1-form}!! A 1-form on $ \mathbb{R}^n$
\begin{itemize}
   \item it takes in a vector and returns a scalar linearly
\end{itemize}

For example, in $ \mathbb{R}^3$, for a vector $ \underline{v}$ 
\[
  \underline{v} = \begin{pmatrix} v_x \\ v_y \\ v_z \end{pmatrix} 
\] 
\begin{align*}
   dx \left( \underline{v} \right) & = v_x  \\
   dy \left( \underline{v} \right) & = v_y  \\
   dz \left( \underline{v} \right) & = v_z  \\
\end{align*}

\textbf{Why is this useful?}: we can do math with it! \\

For example, we could define a 1-form $ \alpha$
\[
  \alpha = dx - 2dy + 5dz
\] 

\begin{align*}
   \alpha \left(  \underline{v} \right) & = dx( \underline{v}) - 2dy ( \underline{v}) + 5 dz (\underline{v})  \\
                                        &= v_x - 2v_y + 5v_z
\end{align*}

We can effectively think of $dx$ as the operator
\[
   dx \sim i \cdot \text{ OR } \begin{pmatrix} 1 \\ 0 \\ 0 \end{pmatrix} \cdot 
\] 

\subsection{1-form fields!}
A 1-form field has a 1-form at every point
\[
  \alpha = y\ dx + x^2\ dy - 3(x^2 + y^2)\ dz
\] 

This is \textbf{very different from a vector field}!!

Consider the scalar field
\[
  f = x^2 - 2xy + y^2
\] 

The gradient (vector field) is 
\[
  \nabla f = \left( 2x-2y \right) \hat{i} + (2y-2x) \hat{j}
\] 
\begin{itemize}
   \item each point nudges in the direction of maximum increase
\end{itemize}

The differential (1-form field) is
\[
  df = \left( 2x-2y \right) dx + (2y-2x) dy
\] 
\begin{itemize}
   \item $dx$ are measurement devices that measures small changes in $x$ 
   \item $df$ can be thought of an operator, which is equivalent to $\nabla f \cdot$
\end{itemize}

\subsection{Something new!!}

\textbf{Given} a 1-form field $\alpha$ and path $\gamma$,  \\

\textbf{Define}
\[
   \int_{\gamma} \alpha
\]  

\textbf{Think!!} What does it mean to take the dot product of the \textbf{velocity vector} along a path and the 1-form at that point?

\[
   \int_{\gamma} \alpha = \int_{\gamma} \left( \alpha_{\gamma(t)} \right) \left( \gamma ' \left( t \right)  \right) dt
\]  

\textbf{Example}

For a 1-form field
\[
  \alpha = y\ dx - 2x\ dy
\] 

and a path
\[
  \gamma(t) = \begin{pmatrix} t \\ t^2 \end{pmatrix} , 0 \leq t \leq 1
\]  

The velocity vector is 
\[
  \gamma ' = \begin{pmatrix} 1 \\ 2t \end{pmatrix} 
\] 

The path integral is 
\begin{align*}
   \int_{\gamma} \alpha &=  \int_{\gamma} y\ dx - 2x\ dy \\
                        &= \int_{t=0}^{1} \left( t^2\ dx - 2t\ dy \right) \cdot \begin{pmatrix} 1 \\ 2t \end{pmatrix}  dt \\
                        &= \int_{t=0}^{1}  t^2 \cdot 1 - 2t \cdot 2t dt \\
                        &= \int_{t=0}^{1}  -3t^2 dt  
\end{align*}

\textbf{Why go through all the trouble?}

This is useful!! The standard definite integral that we know and love
\[
  \int_{x=a}^{b}  f(x) dx 
\] 

is really the integral of the one form field \[
  \alpha = f(x) dx
\] 

over a parameterized path
\[
  \gamma(t) = t, a \leq t \leq b
\] 

Hence, if we follow all the steps outlined above
\[
  \int_{x=a}^{b}  f(x) dx = \int_{t=a}^{b} f(t) dt  
\] 

More concretely....

\textbf{Why integrate 1-form fields?}
\begin{itemize}
   \item Work
   \item Flux
   \item etc etc
\end{itemize}

\section{The APEX}

Consider a gradient (vector field), it has an associated gradient 1-form field. \\

Just as not every vector field is a gradient vector field, not very 1-form field is a gradient 1-form field \\

\begin{framed}
   \textbf{Independence of Path Theorem}: integrating over a gradient 1-form field, the integral is independent of the path and only depends on the endpoints \\

   For a gradient one form field $df$, over a path $\gamma$ from $a$ to $b$

   \[
      \int_{\gamma} df = f( \gamma(b)) - f(\gamma(a))
   \] 

   \textbf{Note} the similarity of this idea to the fundamental theorem of integral calculus
\end{framed}

\section{Thursday Lecture - Recap}

Recall that the biggest themes this week are
\begin{itemize}
   \item 1-form fields in a form similar to
      \[
        \alpha = y^2\ dx - x\ dy
      \] 
   \item integrating over a 1-form field along a parameterized path
\end{itemize}

\subsection{Work, circulation \& Flux}

Consider the planar vector field denoting the force at every point in planar space
\[
  \vec{F} = F_x \hat{i} + F_y \hat{j}
\] 

We can define the work 1-form
\[
   \alpha_{ \vec{F}} = F_x\ dx + F_y\ dy
\] 

Instead of using the unit vectors $ \hat{i}$ and $ \hat{ j}$ to denote the force in each component at every point, we use the 1-form $dx$ and $dy$ to measure the work done in each component. \\

We can also define the flux 1-form
\[
   \Phi_{ \vec{F}} = F_x\ dy - F_y\ dx
\] 

\textbf{Example}

For a parameterized path $\gamma$ along a circle of radius $2$, counter-clockwise (positive) \\

Note that the integral over a 1-form field is independent of the parameterization, only the geometric path \\

Given a vector field in $ \mathbb{R}^2$ 
\[
  \vec{F} = y \hat{i} - x^2 \hat{j}
\] 

We write down the work 1-form
\[
   \alpha_{ \vec{F}} = ydx - x^2 dy
\]  

One possible parameterization of $\gamma$ is
\[
  \gamma(t) = \begin{pmatrix} 2 cos t \\ 2 sin t  \end{pmatrix}, 0 \leq t \leq 2\pi
\] 

The work done is 
\[
   \int_{\gamma} \alpha_{ \vec{F}} = \int_{\gamma } y dx - x^2 dy
\] 

Finding the velocity vector (to relate $dx, dy$ to $dt$
 \[
  \gamma ' = \begin{pmatrix} -2sint \\ 2 cost \end{pmatrix} 
\] 

We can now express work done in terms of $t$ 
\begin{align*}
   \int_{\gamma} \alpha_{ \vec{F}} &= \int_{\gamma } y dx - x^2 dy \\
                                   &= \int_{t = 0}^{2 \pi} (2 sint ) (-2sint) - (4cos^2 t) (2 cost) dt 
\end{align*}

Some trigonometric tricks to remember!!
\begin{itemize}
   \item to integrate $sin^2 t$, use double angle formula
   \item to integrate  $cos^3 t$, pull out a $cos^2 t = (1 - sin^2 t)$, splitting one term into two terms can be integrated individually
\end{itemize}

\section{Independence of path theorem}

The main idea is that for some \textit{special} 1-form fields, the integral is independent of the path \\

More formally, if the 1-form field is a gradient 1-form field, the integral is independent of the path and depends only on the end points
\[
   \int_{\gamma} df = \left. f \right|_{ \gamma(b)}^{\gamma(a)}  = f( \gamma(b)) - f( \gamma(a))
\] 

BUT!!! two complications
\begin{itemize}
   \item gradient 1-forms are rate, when is $\alpha$ a gradient?
   \item how do you find $f$?
\end{itemize}

\subsection{Example}

Given a 1-form field $\alpha$
\[
   \int_{\gamma} \alpha = \int_{\gamma}^{} \left( \frac{2x}{y} - 1 \right) dx + \left( 3y^2 - \frac{x^2}{y^2} \right)  dy
\] 

and a path
\[
  \gamma(t) = \begin{pmatrix} 
    1 + arctan \left( t^2 - t \right)    \\ t + 2 cos 3\pi t
  \end{pmatrix}
\] 

\textbf{Prof-g's nugget of wisdom: the more complicated an integral looks, the more relaxed you should feel.  }

\textbf{Step 1}: look for a potential, $f$ (i.e. $df$ is gradient)

$f$ is a potential if 
\[
  \alpha = df = \frac{\partial f}{\partial x} dx + \frac{\partial f}{\partial y} dy
\] 

We check that
\[
  \frac{2x}{y} -1 = \frac{\partial f}{\partial x}, f = \frac{x^2}{y} - x + C(y)
\] 

We take the partial w.r.t $y$, getting
\[
  \frac{\partial }{\partial y} \frac{x^2}{y} - x + C(y) = -\frac{x^2}{y^2} + C'(y)
\] 

We also check that
\[
   \frac{\partial f}{\partial y} = 3y^2 - \frac{x^2}{y^2} = - \frac{x^2}{y^2} + C'(y) 
\] 

\textbf{ANSATZ}: We try to find $f$ by writing down
\[
  f = \frac{x^2}{y} - x + y^3
\] 

\textit{ANSATZ is the German word for an unverified guess. but it is in German. so it is an intelligent guess. pronounce it like you are shouting at someone -- Prof G}

We verify
\[
  \frac{\partial }{\partial x}  \frac{x^2}{y} - x + y^3 = \frac{2x}{y} - 1
\] 
\[
  \frac{\partial }{\partial y }  \frac{x^2}{y} - x + y^3 = 3y^2 - \frac{x^2}{y^2} 
\] 

Hence, our \textbf{ANSATZ} is no longer an \textbf{ANSATZ} \\

Now the problem is easy, we find the start and end points by
\[
  \gamma(0) = \begin{pmatrix} 1 + arctan(0) \\ 0 + 2 cos 0 \end{pmatrix}   = \begin{pmatrix} 1 \\2 \end{pmatrix} 
\] 

\[
  \gamma(1) = \begin{pmatrix} 1 + arctan(0) \\ 1 + 2 cos 3 \pi \end{pmatrix}   = \begin{pmatrix} 1 \\-1 \end{pmatrix} 
\] 

By independence of path theorem
\begin{align*}
   \int_{\gamma}^{} df &= f(\gamma(b)) - f(\gamma(a)) \\ 
                       &= -1 - \frac{15}{2}  \\
                       &= - 17 / 2
\end{align*}

\subsection{Another example}

Given a 1-form field
\[
   \int_{\gamma}^{}  y e^{xy} dx + \left( xe^{xy} - ze^{-yz} \right) dy + \left( e^z - ye^{-yz} \right) dz
\] 

And a path
\[
  \gamma(t) = \begin{pmatrix} t \\ 2t \\ 3t \end{pmatrix} , 0 \leq t \leq 1
\] 

\textit{You can write down an unsubstantiated ANSATZ, as long as you show that it works!}

Our logic in approaching is problem is

\textbf{If} $ye^{xy} $ is $ \frac{\partial f}{\partial x}$,
\[
   f = e^{xy} + C(y, z)
\] 

\textbf{If} $ xe^{xy} - z e^{-yz} $ is $ \frac{\partial f}{\partial y}$, 
\[
   f = e^{xy} + e^{-yz} + C_2 \left( x, z \right) 
\] 


\textbf{If} $ e^{z} -  ye^{-yz} $ is $ \frac{\partial f}{\partial z}$, 
\[
   f = e^{z} + e^{-yz} + C_3 \left( x, y \right) 
\] 

Merging it all
\[
   f = e^{xy} + e^{-yz} + e^z
\] 

\section{Parameterizing paths}

Make sure you know how to parameterize paths
\begin{itemize}
   \item straight line paths
      \begin{itemize}
         \item for a line from $ \begin{pmatrix} -1 \\ 2 \\ 0 \end{pmatrix} $ to $ \begin{pmatrix}  3 \\ 5 \\ -3 \end{pmatrix} $, we can parameterize it with
            \[
              \gamma(t) = \begin{pmatrix} -1 \\ 2 \\ 0 \end{pmatrix} + t \begin{pmatrix}  3 - (-1) \\ 5 - 2 \\ -3 - 0 \end{pmatrix} , 0 \leq t \leq 1
            \] 
      \end{itemize}
   \item circular paths
\end{itemize}


\section{Prof-g's favorite 1-form}
\[
  \alpha = xdy
\] 

Note that that this is not a gradient
\[
  f = xy, df = ydx + xdy \neq xdy
\] 
 
Is there a way to get from a point in the 3rd quadrant into the 1st quadrant without a movement being detected by  $\alpha$???? \\

Yes, you move in a path parallel to the $x$ axis ($dy = 0$), and then move along the  $y$ axis ($x = 0$), and finally along a path parallel to the axis again. \\

This is not why it is his favorite 1-form, it just helps with checking your understanding of 1-forms






