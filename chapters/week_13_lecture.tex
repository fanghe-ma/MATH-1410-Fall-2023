\chapter{Week 13 lectures}

Recall from before thanksgiving break our discussion of 1-forms. A 1-form is a operator that takes a vector and spits out a scalar. \\

An example of a 1-form field is
\[
  \alpha = xdx - 2ydy
\] 

1-form fields tend to have some nice geometric interpretation when integrated over a path $\gamma$
 \[
    \int_{\gamma} \alpha
\] 

In particular, we touched briefly on the \textbf{independence of path} theorem. \\

Given some gradient 1-form $df$
\[
  \int_{\gamma}^{}   df = \left. f \right|_{\gamma(start)}^{\gamma(end)} 
\] 

\subsection{Thinking more deeply about form fields}

linear 1-forms fields have the form
\[
  \alpha = adx + bdy + cdz
\] 

for some constant $a, b, c$ \\

The basis 1-forms are \[
  dx, dy , dz
\] 

\textbf{Geometric interpretation}: oriented projected length ($dz$ )

linear 2-forms fields have the form
\[
  \beta = dx \wedge dy - 2 dy \wedge dz
\] 

The basis 2-forms are 

\begin{align*}
   dx \wedge dy &= -dy \wedge dx \\
   dy \wedge dz &= -dz \wedge dy \\
   dz \wedge dx &= -dx \wedge dz
\end{align*}

This field takes 2 vectors and spits out a scalar, based on some computation of the projected area

What does $dx \wedge dy$ mean??

\[
   (dx \wedge dy) ( \underline{u}, \underline{v}) = Det \begin{bmatrix} u_x & v_x \\ u_y & v_y
   \end{bmatrix}
\] 

$(dx \wedge dy)$ tells us which components of $u, v$ we will use to compute the determinant

\textbf{Geometric interpretation}: oriented projected area ($dx \wedge dy$ )


\subsection{Generalizing to fields}

Recall for 1-forms, we covered
\[
   \text{Basis 1-forms}(dx, dy) \rightarrow \text{Linear 1-forms} (2dx + 5dy) \rightarrow \text{fields}(ydx + xdy)
\] 

For 2-forms
\[
   \text{Basis 2-forms}(dx \wedge dy) \rightarrow \text{Linear 2-forms} (2dx \wedge dy + 5dy \wedge dz) \rightarrow \text{fields}(x^2dx \wedge dy + ydy \wedge dz)
\] 

\subsection{Algebra of forms}

Very generally, $\wedge \approx$ "wedge" "product"

Consider
\begin{align*}
   &(dx + 2dy) \wedge (dy - 3dz) \\
   =& dx \wedge dy - 3dx \wedge dz + 2 dy\wedge dy - 6 dy \wedge dz
\end{align*}

The rules
\begin{itemize}
   \item $dx \wedge dx = 0$
   \item $dx \wedge dy = - dy \wedge dx$
\end{itemize}

\subsection{Calculus of forms}

The idea of forms is connected to the idea of implicit differentiation. \\

Consider $"d"$ as an operator

\[
   d(\text{0-form field}) = \text{1-form field}
\] 
i.e.
\[
   f \rightarrow df \text{, which is a gradient 1-form}
\] 

For example
\begin{align*}
   f &= x^2 y - z \\
   df &= 2xy dx + x^2 dy - dz
\end{align*}

\begin{framed}
   \textbf{Rule}: 
   \[
     d(f \alpha) = df \wedge \alpha
   \] 
\end{framed}

For example
\begin{align*}
   &d(x^2 dx - xy dy) \\
   =& d(x^2 dx) - d(xy\ dy) \\
   =& (2x\ dx) \wedge dx - (y\ dx + x\ dy) \wedge dy \\
   =& 0 - y\ dx \wedge dy - 0 \\
   =& - y\ dx \wedge dy
\end{align*}

\subsection{Some strange pattern}

A momentary side quest!!!
\[
  d(d \alpha) = 0
\] 

\section{Green's Theorem}

\textbf{In the generic form}

Integrating over some oriented boundary $\partial D$
\[
   \int_{\partial D}^{} fdx + gdy  = \iint_{D} \left( \frac{\partial g}{\partial x} - \frac{\partial f}{\partial y} \right) dA
\] 

\textbf{Work / Circulation}

Given \[
  \vec{F} = F_x \underline{i} + F_y \underline{j}
\] 

\[
  \int_{\partial D}^{} F_x dx + F_y dy = \int_D \left( \frac{\partial F_y}{\partial x} - \frac{\partial F_x}{\partial y} \right)dA
\] 

Where 
\[
   \frac{\partial F_y}{\partial x} - \frac{\partial F_x}{\partial y} = \nabla \times \vec{F} \text{ is known as the \textbf{curl}}
\] 


\textbf{Flux / Divergence}

Given \[
  \vec{F} = F_x \underline{i} + F_y \underline{j}
\] 

\[
  \int_{\partial D}^{} F_x dy - F_y dx = \int_D \left( \frac{\partial F_x}{\partial x} + \frac{\partial F_y}{\partial y} \right)dA
\] 

Where 
\[
   \frac{\partial F_x}{\partial x} + \frac{\partial F_y}{\partial y} = \nabla \cdot \vec{F} \text{ is known as the \textbf{divergence}}
\] 

\section{The end of the story}

\[
  \int_{\partial D}^{}  \alpha = \int_{D}^{} d\alpha  
\] 

Given a 1-form
\[
  \alpha = fdx + g dy
\] 
\begin{align*}
   d\alpha &= df \wedge dx + dg \wedge dy \\
           &= \left( \frac{\partial f}{\partial x}dx + \frac{\partial f}{\partial y}dy \right)  \wedge dx  + \left( \frac{\partial g}{\partial x} dx + \frac{\partial g}{\partial y}dy \right) \wedge dy \\
           &= \frac{\partial f}{\partial y} dy \wedge dx + \frac{\partial g}{\partial x} dx \wedge dy \\
           &= \left(   \frac{\partial g}{\partial x} - \frac{\partial f}{\partial y} \right) dx \wedge dy  \\
           &= \left(   \frac{\partial g}{\partial x} - \frac{\partial f}{\partial y} \right) dA , \text{provided boundary is oriented correctly}
\end{align*}


\section{Continued}

Recall from the previous week the independence of path theorem

\[
  \int_{\gamma}^{ } df = f(\gamma(end)) - f \left(  \gamma \left( start \right)  \right)  
\] 

From this week, we have
\[
   \int_{\partial D}^{} fdx + gdy = \iint_{D}  \left( \frac{\partial g}{\partial x} - \frac{\partial f}{\partial y}  \right) dA 
\] 

\subsection{An example}

Given \[
   \alpha = \left( cos x + 3x^2 - 2y \right) dx + \left( x^3 + 4x - e^{2y} \right) dy
\] 

Integrated over a circle parameterized by
\[
   \gamma(t) = \begin{pmatrix} x_0 + Rcost \\ y_0 + Rsint \end{pmatrix}
\]

It would be horrible to integrate this via usub. But it is much easier with Green's Theorem

\subsection{More on Green's Theorem}

Recall that for
\[
  \alpha = xdy
\] 

\begin{align*}
   \int_{\partial D}^{} xdy &= \iint 1 - 0 dA \\
                            &= A
\end{align*}

Consider applying Green's Theorem to finding the centroid
\begin{align*}
   \overline{X} &= \frac{\iint x dA}{A} = \frac{???????}{\int_{\partial D}^{} x\ dy }
\end{align*}

\[
  \iint_D x dA = \int_{\partial D}^{}  f dx +  gdy= \int_{\partial D}^{ }  \frac{1}{2} x^2 dy 
\] 

Hence 
\begin{align*}
   \overline{X} &= \frac{\iint x dA}{A} = \frac{ \frac{1}{2} x^2 dy}{\int_{\partial D}^{} x\ dy } \\
                &= \frac{1}{2A} \int_{\partial D}^{} x^2 dy 
\end{align*}


\subsection{Another example}
Given 2 0-forms
\[
  f= 3x + 2y - z
\] 

\[
  g = 5y - 4z
\] 

We can compute the gradients
\[
  df = 3dx + 2dy - dz
\] 
\[
  dg = 5dy - 4dz
\] 

What are the gradients of $f, g$?
 \[
  \nabla f = \begin{pmatrix} 3 \\ 2\\ 1 \end{pmatrix}  , \nabla g = \begin{pmatrix} 0 \\ 5 \\ -4 \end{pmatrix} 
\] 

Differentiating the 1-forms gets us
\[
  d(df) = 0 = d(dg)
\] 

Taking the wedge products of $df, dg$,
\begin{align*}
   df \wedge dg &= \left( 3dx + 2dy - dz \right) \wedge \left( 5dy - 4dz \right)  \\
                &= 15 dx \wedge dy - 12 dx \wedge dz - 8 dy \wedge dz - 5dz \wedge dy \\
                &= 15 dx \wedge dy - 12 dx \wedge dz - 3 dy \wedge dz
\end{align*}


\section{Flux 2-forms}

A 3-D Vector field has associated with it a flux 2-form
\[
   \phi_{\vec{F}} = F_x dy \wedge dx + F_y dz \wedge dx + F_z dx\wedge dy
\] 

The wedge product of 2 scalar fields $f$ and $g$ gives the flux 2-form of a constant vector field, i.e.

If
\[
   \phi_{\vec{F}} = 15 dx \wedge dy - 12 dx \wedge dz - 3 dy \wedge dz
\] 

Then
\[
  \vec{F} = \begin{pmatrix} -3 \\ 12 \\ 15 \end{pmatrix}  = \begin{pmatrix} 3 \\ 2 \\ -1 \end{pmatrix}  \times \begin{pmatrix} 0 \\ 5 \\ -4 \end{pmatrix} 
\] 

\section{Curl, grad, divergence}

In 3D, there are two ways to interpret a vector field
\[
  \vec{F} = F_x \underline{i} + F_y \underline{j} + F_x \underline{k}
\] 

This can be interpreted as 
\[
   \alpha_{\vec{F}} \text{1-form}
\] 
\[
   d \alpha_{\vec{F}} \text{ is known as the curl, } \nabla \times \vec{F}\text{, measures rotation, infinitesimal spin}
\] 

This can also be interpreted as 
\[
   \phi_{\vec{F}} \text{2-form}
\] 
\[
   d \phi_{\vec{F}} \text{ is known as the div, } \nabla \cdot \vec{F} \text{, measures expansion, infinitesimal flux}
\] 

\subsection{Example}

Given
\[
  \vec{F} = x \underline{i} + xz \underline{j} + yz \underline{k}
\] 

For divergence
\begin{align*}
   Div( \vec{F}) = \nabla \cdot \vec{F} &= \frac{\partial }{\partial x}x + \frac{\partial }{\partial y} \left( -xz \right)  + \frac{\partial }{\partial z} \left( yz \right)  \\
   &= 1 + y
\end{align*}

For curl
\begin{align*}
   \nabla \times \vec{F} = \left( z + x \right) \underline{i} + \left( 0 \right) \underline{j} + \left( -z \right) \underline{k}
\end{align*}

\section{3 non-trivial observations}
1. "d" is curl (differentiating a 1-form)
\[
   d \alpha_{\vec{F}} = \phi_{\nabla \times \vec{F}} 
\] 

2. "d" is divergence (differentiating a 2-form)
\[
   d \phi_{\vec{F}} = \left( \nabla \times \vec{F} \right) dx \wedge dy \wedge dz
\] 

3. "d" is gradient (differentiating a 0-form)
\[
   df = \alpha_{\nabla f}
\] 

