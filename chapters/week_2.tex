\chapter{Week 2}

\section{The Dot Product}
The dot product for two vectors $\underline{u}$ and $\underline{v}$ in $\mathbb{R}^n$ separated by angle $\theta$ is defined as
\begin{framed}
   \[
     \underline{u} \cdot \underline{v} = \lVert \underline{u} \rVert \lVert \underline{v} \rVert cos \theta
   \] 
\end{framed}

For two vectors, $\underline{u} =  \begin{pmatrix} u_1 \\ u_2 \\u_3\\ \vdots u_n\end{pmatrix}$ and $ \underline{v} = \begin{pmatrix} v_1 \\ v_2 \\ v_3  \vdots \\ u_n \end{pmatrix}$ their dot product is computed as
\begin{framed} 
   \[
     \underline{u} \cdot \underline{v} = u_1 v_1 + u_2 v_2 + u_3 v_3 + \hdots + u_n v_n
   \] 
\end{framed}

Properties of the dot product
\begin{framed}
The dot product has the following properties
   \begin{itemize}
      \item Commutativity: $ \underline{u} \cdot \underline{v} = \underline{v} \cdot \underline{u}$ 
      \item $ \underline{u} \cdot \underline{0} = 0$ 
      \item $\underline{v} \cdot \underline{v} = \lVert \underline{v} \rVert ^2$
   \end{itemize}
   Note that the dot product of two vectors is a \emph{scalar}
\end{framed}

\subsection{Orthogonality and the dot product}
\begin{framed}
For two vectors $ \underline{u}$ and $ \underline{v} \neq  \underline{0}$
   \begin{align*}
       \underline{u} &\perp \underline{v}  \\
                     &\Updownarrow \\
      cos \theta &= 0 \\
                 &\Updownarrow \\
      \underline{u} \cdot \underline{v} &= 0
   \end{align*}
\end{framed}

\subsection{Orthogonal Projection}
\begin{framed}
For two vectors $ \underline{v}$ and $ \underline{u}$, the length of $ \underline{v}$ along the $ \underline{u}$ axis is 
   \[
      \frac{\underline{v} \cdot   \underline{u}}{ \lVert \underline{u} \rVert } 
   \] 
\end{framed}

\subsection{Implicit definition of hyperplanes using the dot product}
\begin{framed}
   The hyperplane passing through a point $\underline{x}_0$ and orthogonal to vector $\underline{n}$ is 
   \[
      \left( \underline{x} - \underline{x}_0 \right) \cdot \underline{n} = 0
   \] 
\end{framed}

\section{The Cross Product}
The cross product of two vectors $ \underline{u}$ and $ \underline{v}$ is 
\begin{framed}
  \[
    \underline{u} \times \underline{v} = 
      \begin{pmatrix}    
         u_2 v_3 - u_3 v_2 \\
         u_3 v_1 - u_1 v_3 \\
         u_1 v_2 - u_2 v_1
      \end{pmatrix}
  \] 
\end{framed}

\subsection{Properties of the cross product}
\begin{framed}
  The cross product has the following properties
  \begin{itemize}
     \item Anti-commutativity: $\underline{u} \times \underline{v} = - \underline{v} \times \underline{u}$
     \item Mutual orthagonality: $\underline{u} \times \underline{v} \perp \underline{u}$
     \item $\underline{u} \times \underline{0} = \underline{0}$
     \item $\underline{v} \times \underline{v} = - \underline{v} \times \underline{v} = 0$
  \end{itemize}
\end{framed}

\subsection{Geometric understanding of the cross product}
\begin{framed}
  For two vectors $ \underline{u}$ and $ \underline{v}$ separated by angle $\theta$ 
  \[
    \lVert \underline{u} \times \underline{v} \rVert = \lVert \underline{u} \rVert \lVert \underline{v} \rVert sin \theta
  \] 
\end{framed}

\subsection{Distance to a line and the cross product}
\begin{framed}
   The distance from a point $P$ to a line containing point $Q$ and parallel to the vector $ \underline{v} is
   \frac{ \lVert \vec{QP} \times \underline{v} \rVert }{ \lVert \underline{v} \rVert }$
\end{framed}

\subsection{The Scalar Triple product}
\begin{framed}
The volume of the parallelopiped generated by vectors $ \underline{u}$, $ \underline{v}$ and $ \underline{w}$ is
\[
  \lvert \underline{u} \cdot \left( \underline{v} \times \underline{w} \right) \rvert
\] 

Note that 
\[
  \lvert \underline{u} \cdot \left( \underline{v} \times \underline{w} \right) \rvert =
  \lvert \underline{v} \cdot \left( \underline{w} \times \underline{u} \right) \rvert =
  \lvert \underline{w} \cdot \left( \underline{u} \times \underline{v} \right) \rvert =
\]
\end{framed}


\section{Intro to vector calculus}

For a curve in 3-D space $\gamma : \mathbb{R} -> \mathbb{R}^3$
   \begin{center}
      \begin{tabular}{|c|c|c|}
         \hline
         Position & Velocity & Acceleration \\ 
         \hline
         $\gamma (t) = \begin{pmatrix} x(t) \\ y(t) \\ z(t)   \end{pmatrix}$ &
         $\gamma^{\prime} = \begin{pmatrix} x^{\prime} \\ y^{\prime} \\ z^{\prime}   \end{pmatrix}$ &
         $\gamma^{\prime \prime} = \begin{pmatrix} x^{\prime \prime} \\ y^{\prime \prime} \\ z^{\prime \prime}   \end{pmatrix}$ \\ 
         \hline
         $ \underline{r}$ & $ \underline{v}$ & $ \underline{a}$ \\
         \hline
      \end{tabular}
   \end{center}

\subsection{Derivative of a vector}
\begin{framed}
   \[
   \underline{v} = \underline{r}^{\prime} = \lim_{h \to 0} \frac{ \underline{r}(t+h) - \underline{r}(t)}{h}
   \] 
\end{framed}

\subsection{Arclength of a parameterized curve}
\begin{framed}
   Given a parameterized curve $\gamma : \mathbb{R} \rightarrow \mathbb{R}^n$
   \begin{align*}
      l &= \int dl\\
      dl &= \lVert \gamma^{\prime} \rVert dt
   \end{align*}
   Hence
   \begin{align*}
      l &= \int \lVert \gamma^{\prime} \rVert dt
 \\
   \end{align*}

\end{framed}

\subsection{Rules of vector differentiation}
\begin{framed}
  For vector functions $ \underline{u}(t)$ and $ \underline{v}(t)$ 
  \begin{align*}
     \left( \underline{u} \cdot \underline{v} \right)^{\prime} 
     &= \underline{u}^{\prime} \cdot \underline{v} + \underline{u} \cdot \underline{v}^{\prime}\\
     \left( \underline{u} \times \underline{v} \right)^{\prime} 
     &= \underline{u}^{\prime} \times \underline{v} + \underline{u} \times \underline{v}^{\prime}
  \end{align*}
\end{framed}

\section{Vectors \& Physical Motion}

\subsection{Finding the velocity and position vector with acceleration}

\begin{framed}
   For a particle with position vector $ \underline{r}(0) = \begin{pmatrix}  2 \\ 4 \\ -1  \end{pmatrix} $ and velocity vector $ \underline{v}(0) = \begin{pmatrix}  -1 \\0 \\2  \end{pmatrix} $ and acceleration vector $ \underline{a}(t) = \begin{pmatrix}  -cos(t) \\ 3e^{-t} \\ 6t  \end{pmatrix} $ 

   The velocity vector at time $t$ is
      \begin{align*}
         \underline{v}(t) &= \int_{0}^{t} \underline{a}(t)dt + \underline{v}(0)\\
                          &= \int_{0}^{t} 
                          \begin{pmatrix}  
                             -cost \\
                             3e^{-t}\\
                             6t \\
                           \end{pmatrix} + \underline{v}(0)\\
                          &= \begin{pmatrix} 
                             -sint \\
                             -3e^{-t}\\
                             3t^2
                          \end{pmatrix}dt + \underline{v}(0)\\
                          &= \begin{pmatrix} 
                            -sint - 1\\
                            -3e^{-t} - 3 \\
                            3t^2 +2
                          \end{pmatrix}
      \end{align*}

  The position vector at time $t$ is
  \begin{align*}
     \underline{r}(t) &= \int_{0}^{t} \underline{v}(t)dt + \underline{r}(0)\\
                       &= \int_{0}^{t}\begin{pmatrix} 
                         -sint - 1\\
                         -3e^{-t} - 3 \\
                         3t^2 +2
                       \end{pmatrix} dt + \underline{r}(0) \\
                       &= \begin{pmatrix} 
                         cost - t - 1\\
                         -3e^{-t} + 3t -3 \\
                         t^3 + 2t
                       \end{pmatrix} + \begin{pmatrix} 
                         2 \\4 \\ 1  
                       \end{pmatrix}
  \end{align*}
\end{framed}

\subsection{Unit tangent and normal vectors}

The velocity and acceleration vectors at time $t$ can be expressed as a combinations of two orthogonal vectors.

\begin{framed}
   $ \underline{v}(t)$ and $ \underline{a}(t)$ can be expressed in terms of
   \begin{itemize}
      \item $ \underline{T}$, \emph{unit tangent} vector to $ \underline{v}$ at time t
      \item $ \underline{N}$, \emph{unit normal} vector to $ \underline{v}$ at time t
   \end{itemize}

   where 
   \begin{align*}
      \underline{T}(t) &= \frac{ \underline{v}(t)}{ \lVert \underline{v} \rVert } \\
      \underline{N}(t) &= \frac{ \underline{T}^{\prime}(t)}{ \lVert \underline{T}^{\prime} \rVert } \\
   \end{align*}
\end{framed}

The acceleration vector $a$ T can be decomposed as
\begin{framed}
   \begin{align*}
      \underline{a} &= a_T \underline{T} + a_N \underline{N} \\
                    &= \left( \frac{d}{dt} \lVert \underline{v} \rVert \right) + \left( \kappa \lVert \underline{v} \rVert^2 \right) \underline{N} 
   \end{align*}

   where \begin{align*}
      \kappa(t) &= \lVert \frac{d}{dl} \underline{T}dl \rVert = \frac{ \lVert \frac{d}{dt} \underline{T}(t) \rVert }{ \frac{dl}{dt}}
 \\
                &= \frac{ \lVert \underline{v} \times \underline{a} \rVert }{ \lVert \underline{v} \rVert^3} \text{ (in 3-D)}
   \end{align*}
\end{framed}

\subsection{The osculating circle}

The osculating circle, or \emph{best fit circle} lies in the normal-tangent plane, and has the reciprocal of the curvature as its radius

\begin{framed}
  The osculating center is given by \[
    \underline{r}(t) + \frac{1}{\kappa(t)} \underline{N}(t)
  \] where $\kappa (t)$ is the radius
\end{framed}

\subsection{The unit binormal vector}

The degree of \emph{twist} out of the osculating plane is given by the unit binormal vector
\begin{framed}
   For a curve $\gamma : \mathbb{R} -> \mathbb{R}^n$, the binormal is
   \[
     \underline{B} = \underline{T} \times \underline{N}
   \] 

   The strength of the twisting is given by \[
      \tau(t)= - \underline{N}(t) \cdot \frac{d}{dl} \underline{B}(l)
   \] 
\end{framed}







