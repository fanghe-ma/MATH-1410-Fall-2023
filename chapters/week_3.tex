\chapter{Week 3}

\section{Matrix Equations}

\subsection{Linear System of Equations}

\begin{framed}
   A linear system of equations, such as 
   \begin{align*}
      x - 3y + 3z &= 8 \\
      5x + y - 2z &= 7 \\
      -2x + y -z &= -6
   \end{align*}

   Can be expressed as \[
     A \underline{x} = \underline{b}
   \] 

   Where $A$ is a matrix $
     A =  \begin{bmatrix} 
           1 & -3 & 3 \\  
           5 & 1 & -2 \\  
           -2 & 1 & -1
     \end{bmatrix}
      $, $ \underline{x} = \begin{pmatrix} x \\ y \\ z \end{pmatrix} $ and $ \underline{b} = \begin{pmatrix} 8 \\ -7 \\ 6 \end{pmatrix} $
   
     The system has a unique solution where \[
       \underline{x} = \begin{pmatrix} 2 \\ -1 \\ 1 \end{pmatrix} 
     \] 
\end{framed}

\section{Gaussian elimination, or row reduction}

\begin{framed}
   For a system of linear equations \[
      \begin{bmatrix} 
            1 & 2 & 0 & 3 & 0\\
            0 & 0 & 0 & 2 & 1\\
            0 & 0 & 6 & 4 & -2\\
            0 & 2 & 0 & -1 & 4\\
            0 & 0 & 0 & 0 & 2\\
      \end{bmatrix} 
      \begin{pmatrix} x_1 \\ x_2 \\ x_3 \\ x_4 \\ x_5 \end{pmatrix} = \begin{pmatrix} 
       0 \\ -2 \\ 4 \\ 3 \\ 10
    \end{pmatrix} 
   \] 
   \\

   We can convert it into the augmented matrix 
   \[ 
      \left[ 
         \begin{array}{c c c c c | c}
            1 & 2 & 0 & 3 & 0 & 0 \\
            0 & 0 & 0 & 2 & 1 & -2 \\
            0 & 0 & 6 & 4 & -2 & 4 \\
            0 & 2 & 0 & -1 & 4 & 3 \\
            0 & 0 & 0 & 0 & 2 & 10 \\
        \end{array}
      \right] 
   \] 

   The system is unchanged under the following operations
   \begin{itemize}
      \item Switching 2 rows \[
        R_m \leftrightarrow R_n
      \] 

      \item Recaling a row \[
         m \times R_1, m \in \mathbb{R}
      \] 

      \item Adding a multiple of one row to another \[
            R_2 - \frac{1}{2}R_1
      \] 
   \end{itemize}

   Given these 3 operations, for each column $j$ in matrix $A$, 
   \begin{enumerate}
      \item Swap  $R_i$ with any row $R_n$ provided $n > i$ such that the value  $A_{ij}$ is non-zero
      \item Scale $R_i$ such that $A_{ij}$ is $1$ or $-1$
      \item Clear values in column $j$ below $A_{ij}$ by adding a multiple of $R_i$
      \item Repeat for the next $j$
   \end{enumerate}
\end{framed}


\section{Inverse matrices}

\begin{framed}
   The \textbf{inverse} of $A$ is a matrix $A^{-1}$ which satisfies  \[
      A A^{-1} = I = A^{-1} A
   \] 
\end{framed}

\subsection{Inverse 2-by-2 matrices}

\begin{framed}
   For a 2-by-2 matrix $A = \begin{bmatrix} a & b \\ c & d \\ \end{bmatrix} $, the inverse is \[
   A^{-1} = \frac{1}{ad-cb} \begin{bmatrix} 
      d & -b \\ -c & a  
   \end{bmatrix} \text{, provided $ab-dc \neq 0$}
   \] 
\end{framed}

\subsection{Uniqueness of the inverse}
\begin{framed}
   \textbf{Claim}: The inverse of a matrix is unique if it exists \\

   If $A$ has multiple inverses $B$ and $C$, then by definition \[
     AB = I = BA, AC = I = CA
   \] 

   By associative property, \[
     B = IB = (CA)B = C(AB) = CI = C
   \] 
\end{framed}

\subsection{Solving a system of linear equations with the inverse}
\begin{framed}
   Given a system $A \underline{x} = \underline{b}$, applying the inverse to both sides gives \[
      A^{-1} A \underline{x} = \underline{x} = A^{-1} \underline{b}
   \] 
\end{framed}

\subsection{Finding the inverse via Gaussian Elimination or row reduction}

\begin{framed}
   Given $A = \begin{bmatrix} 1 & 3 & 1 \\ 2 & 3 & -1 \\ -4 & 0 & 2 \end{bmatrix} $,the inverse $A^{-1}$ can be found via row reduction, \[
   \left[  
      \begin{array}{c c c | c c c }
         1 & 3 & 1 & 1 & 0 & 0 \\
         2 & 3 & -1 & 0 & 1 & 0 \\
         4 & 0 & 2 & 0 & 0 & 1 \\
      \end{array} \right]
      \rightarrow 
      \left[
      \begin{array}{c c c | c c c }
         1 & 0 & 0 & \frac{1}{3} & -\frac{1}{3} & - \frac{1}{3} \\
         0 & 1 & 0 & 0 & \frac{1}{3} & \frac{1}{6} \\
         0 & 0 & 1 & \frac{2}{3} & -\frac{2}{3} & - \frac{1}{6} \\
      \end{array}
   \right] 
   \] 

  
\end{framed}

\subsection{Finding the inverse of a block diagonal matrix}

\begin{framed}
   For a matrix  \[
     A = \begin{bmatrix} 
        A_1 & \hdots & \hdots \\
        \vdots & A_2 & \vdots \\
        \vdots & \hdots & A_3 \\
     \end{bmatrix}
   \] 
   \[
      A^{-1} = \begin{bmatrix} 
        A_1^{-1} & \hdots & \hdots \\
        \vdots & A^{-1}_2 & \vdots \\
        \vdots & \hdots & A^{-1}_3 \\
     \end{bmatrix}
   \] 
\end{framed}





