\chapter{Week 10}

\section{Centroids \& Centers of mass}

\subsection{Centroids in 2D}

\begin{framed}
   The centroids of a region between graphs can be found using double integrals and averages

   \begin{align*}
      \overline{x} &= \frac{1}{A} \iint_{R} x\ dA  \\
                   &= \frac{1}{A} \int_{x=a}^{b} \int_{y=g(x)}^{f(x)} x\ dy\ dx
   \end{align*}
  
   \begin{align*}
      \overline{y} &= \frac{1}{A} \iint_{R} y\ dA  \\
                   &= \frac{1}{A} \int_{x=a}^{b} \int_{y=g(x)}^{f(x)} y\ dy\ dx
   \end{align*}
\end{framed}

\subsection{Centroids in 3D}
\begin{framed}

   The centroid of a region $R$ in $ \mathbb{R}^n$ is the point $ \overline{ \underline{x}} $ with coordinates
   The $i-th$ component of the centroid coordinates in $x_1, x_2 \hdots x_n$ is
   \begin{align*}
      \overline{x_i} &= \frac{1}{V} \int_R x_i d \underline{x} \\
                     &= \frac{\int_R x_i d \underline{x} }{\int_R 1 d \underline{x}}
   \end{align*}
\end{framed}

\subsection{Center of Mass}

\begin{framed}
   The \textbf{center of mass} of a region $R$ in $ \mathbb{R}^n$ with mass density $ \rho ( \underline{x})$ is the point $ \overline{ \underline{x}}$ with coordinates
   \begin{align*}
      \overline{x_i} &= \frac{
         \int_R x_i \rho( \underline{x}) d \underline{x} 
      }{
         \int_R \rho ( \underline{x}) d \underline{x}
      } \\
      &= \frac{
         \int_R x_i dM
      }{
         \int_R dM
      } \\
      &= \frac{1}{M} \int_R x_i dM
   \end{align*}
\end{framed}

\section{Moments of inertia}

\begin{framed}
   Let $R \in \mathbb{R}^n$ be a solid body to be rotated about an axis

   The moment of inertia measures the resistance to rotation, where
   \[
     I = \int_R dI
   \] 

   and \[
     dI = r^2 dM
   \] 
\end{framed}

\subsection{Parallel Axis Theorem}

\begin{framed}
   If $I_0$ is the moment of inertia of an object about an axis through its center of mass, then the moment of inertia about a parallel axis $D$ away equals
   \[
     I_D = I_0 + MD^2
   \] 
\end{framed}

\subsection{Radius of Gyration}

\begin{framed}
   Let  $ \mathbb{R} \in \mathbb{R}^n$ be a solid body to be rotated about an axis. If all the mass were concentrated at a single point, the radius of gyration is the radius at which the concentrated mass would have the same moment of inertia

   \[
     I = M r_g^2
   \] 
   \[
      r_g = \sqrt{ \frac{I}{M}}
   \] 
  
\end{framed}

\section{Inertia Matrix}
\section{Solid Body Mechanics}
\section{Probability \& Integration}

For a random $ \underline{X}$ taking on values in $R^n$ \\

The probability density function is 
\[
  \rho : \mathbb{R}^n \rightarrow \mathbb{R}
\] 

such that
\[
  \rho \geq 0
\] 

and 
\[
   \int_{ \mathbb{R}^n}  \rho ( \underline{x}) d \underline{x} = 1
\] 

The probability element is given by
\[
  dP = \rho ( \underline{x }) d \underline{x}
\] 

Hence the probability of $P( \underline{X} \in A)$ is 
 \[
  P( \underline{X} \in A) = \int_A dP = \int_A \rho ( \underline{x}) d \underline{x}
\] 

The expectation $E ( \underline{X}) $ is the $\rho$ weighted centroid 
\[
   E( \underline{X}) = \int_{ \mathbb{R}^n} \underline{x} \rho ( \underline{x}) d \underline{x}
\]  


\section{Independence \& Covariance}
\section{Covariance Matrices}
