\chapter{Week 2 Lecture 2}

\section{Finding the angle between the grand diagonal and an adjacent vector of a unit cube}
\begin{framed}
   In $3-D$, the adjacent vector is $ \begin{pmatrix} 0 \\ 0 \\ 1 \end{pmatrix} $ and the grand diagonal is $ \begin{pmatrix} 1 \\ 1 \\ 1 \end{pmatrix} $, the angle $ \theta $ is given by \[
   arccos \frac{1}{\sqrt{3}}
   \] 
  
   In $4-D$, the adjacent vector is $ \begin{pmatrix} 0 \\ 0 \\ 1 \end{pmatrix} $ and the grand diagonal is $ \begin{pmatrix} 1 \\ 1 \\ 1 \end{pmatrix} $, the angle $ \theta $ is given by \[
   arccos \frac{1}{\sqrt{4}}
   \] 

   In the limit of $ n \to \infty$, \[
      \frac{1}{\sqrt{n}} \to 0
   \] 
\end{framed}



\section{Finding the arclength of a parameterized curve}

\begin{framed}
   For a curve parameterized by position vector $\gamma$ 
   \[
     \gamma(t) = \begin{pmatrix}
        R_1 cost \\ R_1 sint \\ R_2 cost  \\ R_2 sint
       
     \end{pmatrix}
   \] for $0 \le t \le 2\pi$

   \begin{align*}
      l &= \int dl \\
        &= \int_0 ^{2\pi} \lVert \gamma ^{\prime} \rVert dt \\
        &= \int_0 ^{2\pi} \lVert \begin{pmatrix} 
        -R_1 sint \\ R_1 cost \\ -R_2 sint  \\ R_2 cost
        \end{pmatrix}
     \rVert dt \\
        &=2 \pi \sqrt{R_1^2 + R_2^2} \\
   \end{align*}

   Hence, generalizing for a curve in $2n$ dimensions
   \[
     \gamma(t) = \begin{pmatrix}
        R_1 cost \\ R_1 sint \\ R_2 cost  \\ R_2 sint \\ \vdots \\ R_n cost \\ R_n sint
     \end{pmatrix}
   \] for $0 \le t \le 2\pi$

   The arclength $l$ is given by 
   \begin{align*}
      l &= 2\pi \sqrt{R_1^2 + R_2^2 + \hdots + R_n^2} \\
        &= 2 \pi \sqrt{ \frac{1}{6} (n) (n+1) (2n+1)}
   \end{align*}
\end{framed}

\section{Proving that the unit normal vector is orthogonal to unit tangent}
\begin{framed}
   A curve $ \gamma(t)$ has unit tangent vector $ \underline{T} = \frac{ \gamma ^{\prime}(t)}{ \lVert \gamma ^{\prime}(t) \rVert } $ and unit normal vector $ \underline{N} = \frac{ \frac{d}{dt} \underline{T}}{ \lVert \frac{d}{dt} \underline{T} \rVert }$ 

   Claim:
   \begin{align*}
      \underline{N} &\perp \underline{T}
   \end{align*}
   

   Proof:
   \begin{align*}
      \underline{T} \cdot \underline{T} ^{\prime} &= 0 \text{ since } \\
      \left( \underline{T} \cdot \underline{T} \right) ^{\prime} &= \underline{T} ^{\prime} \cdot \underline{T} + \underline{T} \cdot \underline{T} ^{\prime} \text{ by product rule}\\
                                                                 &= 2 \underline{T} ^{\prime} \cdot \underline{T} \text{ by commutative property of the dot product, and} \\ 
      \left( \underline{T} \cdot \underline{T} \right) ^{\prime} &= 0
   \end{align*}
\end{framed}


