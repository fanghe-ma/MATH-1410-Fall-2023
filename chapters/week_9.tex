\chapter{Integration}

Recall from single variable that there were two different kinds of integrals

\begin{tabular}{|r| c| c|}
   \hline
   & Definite & Indefinite \\
   \hline
   Notation & $ \int_{x = a}^{b} f(x) dx $ & $ \int_{}^{} f(x) dx) $\\ 
   \hline
   Result & Scalar value & Class of functions \\
   \hline
   Definition & A limit of Riemann Sums & An antiderivative \\
   \hline
   Application & Used to compute area, work, force, etc & Used to compute definite integrals \\
   \hline
\end{tabular}
\\
For this course, we will focus on definite integrals. \\

Recall that the single variable definite integral is defined as 
\begin{framed}
  

\[
   \int_{x=a}^{b}  f(x) dx = \lim \limits_{\triangle x \to 0^+} \sum \limits_{i} f(x_i) \triangle x 
\] 
\end{framed}

The two objects are joined by the Fundamental Theorem of Integral Calculus 
\begin{itemize}
   \item From the indefinite integral, find the antiderivative, evaluate at the end points and take the difference
   \item You could think of the indefinite integral as a \textbf{means to an end}
\end{itemize}

In multivariable calculus, there is \textbf{no longer the indefinite integral}
\begin{itemize}
   \item The indefinite integral has no multivariate analogue
\end{itemize}

There are certain 1-D interpretations of the derivative and integral that do not persist into higher dimensions. For example
\begin{itemize}
   \item The derivative as \textit{the slope}
   \item The integral as \textit{the area}
\end{itemize}

Instead, we work with the intuition 
\begin{itemize}
   \item The derivative as a \textbf{linear transformation}
   \item The integral as a \textbf{mass} \\
\end{itemize}



In the multivariate case, we use the following notation
\begin{framed}
  For $f: \mathbb{R}^n \rightarrow R$ the integral is defined as
  \[
    \int_{R}^{}  d \underline{x}  
  \] 
  OR
  \[
     \int_{}^{} \hdots \iint_{R} f dx_1\ dx_2\ \hdots dx_n
  \] 
\end{framed}

\section{Computing integrals and the Fubini theorem}

\begin{framed}
   \textbf{Theorem}: Some complicated stuff

   \textbf{Idea}: For some function which depends on a bunch of variables
   \[
     f( \underline{x}) d \underline{x} 
   \]  
   where 
   \[
   \underline{x} = (x_1, x_2,  \hdots x_n)
   \] 
   \[
   d\underline{x} = dx_1, dx_2,  \hdots dx_n
   \] 

   The integral over $ \mathbb{R}^n$ with respect to the volume element $d \underline{x}$ is
   \[
      \int_{ \mathbb{R}^n} f ( \underline{x}) d \underline{x} =
        \int \left( 
           \hdots \left( 
              \int \left( 
                 \int f dx_1
              \right) dx_2
           \right) 
        \right)  dx_n
   \]  

   Most importantly: the order in which the integral is computed does not matter
  
\end{framed}

\section{Double \& Triple Integrals}

\begin{framed}
   A double integral is given by
   \[
      \iint_{R} f\ dA = \iint_{R} f(x,y)\ dx\ dy
   \]  

   A triple integral is given by
   \[
      \iiint_{R} f\ dV = \\\int_{R} f(x, y, z)\ dx\ dy\ dz
   \] 
\end{framed}

\subsection{Finding the area in 2 dimensions}
To find the area, we integrate the area element $dA$, 
\[
  A = \int dA
\] 
Where $dA$, or the area of an infinitesimal rectangle, is given by
 \[
  dA = dx\ dy = dy\ dx
\] 
Therefore, 
\[
  A = \int dA = \iint 1 dx\ dy = \iint 1 dy\ dx
\] 

\subsection{Changing the limits}
In general, to figure out the limits of integration, we
\begin{enumerate}
   \item choose a variable to integrate with respect to, such as $dy$
   \item \textbf{fix $x$}, i.e. for a given value of $x$, read the top and bottom boundaries, which are some functions of $y = f(x)$ \\
   \item determine if integral needs to be split into the sum of multiple integrals 
   \item move on to the next variable
\end{enumerate}


\section{Averages}

For the classical case, in 0-dimensions
\[
   f: {1, 2, \hdots , n} \rightarrow \mathbb{R}
\] 

$f$ maps discrete values to a point each, the average is
\[
   \overline{f} = \frac{1}{n} \sum \limits_{i=1}^{n} f_i
\] 

In 1-dimension
\[
  f: [a,b] -> \mathbb{R} 
\] 
$f$ maps values between the boundaries $a, b$ to a value, the average is \[
   \overline{f} = \frac{1}{b-a} \int_{a}^{b}  f 
\] 

In the multivariate case
\[
   \overline{f} = \frac{ \int_{R}^{} f d \underline{x} }{ \int_{R}^{} d \underline{x}  } = \frac{1}{VOL_n (R)} \int_{R}^{} f d \underline{x} 
\] 

