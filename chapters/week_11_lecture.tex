\chapter{Week 11 lectures}

\section{Changing coordinates}

\subsection{Polar coordinates}

Consider a circle centered at origin with radius $R$, to compute the area using $x-y$ coordinates, we can evaluate
\[
   A = \int_{x = -R}^{R} \int_{y = -\sqrt{R^2 - x^2}}^{\sqrt{R^2 - x^2}} dy dx 
\] 

We could instead convert to polar coordinates, where
\[
  dA = r\ dr\ d \theta
\] 

\begin{align*}
   A &= \int_{x = -R}^{R} \int_{y = -\sqrt{R^2 - x^2}}^{\sqrt{R^2 - x^2}} dy dx  \\
     &= \int_{ \theta = 0 }^{2 \pi} \int_{r = 0}^{R} dr\ d \theta  \\
     &= \left. r \right|_{0}^{R} \left. \theta \right|_{0}^{2 \pi} \\
     &= 2 \pi R
\end{align*}

\subsection{Cylindrical coordinates}
\[
  dV = r\ drd \theta dz
\] 

\begin{align*}
   x &= r cos \theta \\
   y &= r sin \theta \\
   z &= z
\end{align*}

Considering finding the moment of inertia of a cylinder of height $h$ and radius $R$ about the $z$ axis

\begin{align*}
   I &= \int dI \\
     &= \int r^2 dM \\
     &= \int_{z= 0}^{h} \int_{ \theta = 0}^{2 \pi}  \int_{r=0}^{R} r^2 \left( r dr d \theta d z \right)  \\
     &= \frac{2\pi h R^4}{4}
\end{align*} 

\subsection{Spherical coordinates}

The spherical coordinates are 
\begin{align*}
   x &= \rho cos \theta sin \phi \\ 
   y &= \rho sin \theta sin \phi \\
   z &= \rho cos \phi \\
\end{align*}

Where
\begin{itemize}
   \item $\rho$ : $0 \leq \rho \leq R$, radius
   \item $\theta$ : $0 \leq \theta \leq 2 \pi$, azimuth
   \item $\phi$ : $0 \leq \phi \pi R$, inclination (angle from north pole)
\end{itemize}

What is the volume element in spherical coordinates? is it perhaps
\[
  V = \int dV = \int_{ \phi = 0 }^{ \pi}  \int_{ \theta}^{2 \pi} \int_{ \rho = 0}^{R}   d \rho d \theta d \phi
\]  

\textbf{NO!!} The volume element is 
\[
  dV = \rho^2 sin \phi\ d \rho\ d \phi\ d \theta
\] 

We can try computing the volume of a sphere to test it out
\begin{align*}
   V &= \int dV \\
     &= \int_{ \phi = 0 }^{ \pi}  \int_{ \theta}^{2 \pi} \int_{ \rho = 0}^{R}  \rho^2 sin \phi  d \rho d \theta d \phi \\
     &= \frac{R^3}{3} 2 \pi  \times 2 \\
     &= \frac{4}{3} \pi R^3
\end{align*}

\subsection{Finding the area element in polar coordinates}

We want to show that \[
  dA = r\ dr \ d \theta
\] 

Recall linear transformations from earlier in the semester, and the relationship between the determinant and the scaling of n-volume\\

Consider the transformation into polar coordinates
\[
  P \begin{pmatrix}  r \\ \theta \end{pmatrix} = \begin{pmatrix}  x \\ y \end{pmatrix}  = \begin{pmatrix}  r cos \theta \\ r sin \theta \end{pmatrix} 
\] 

Recall from the chapter on differentiation that 
\[
  \left[ D P \right]_{}  = \begin{bmatrix} 
     cos \theta & -r sin \theta \\ sin \theta & rcos \theta  
  \end{bmatrix}
\] 

Hence, 
\[
  det \left[ D P \right]_{}  = r \left( cos^2 \theta + sin^2 \theta \right)  = r
\] 

And
\[
  dx dy = r\ dr\ d \theta
\] 

\subsection{Finding the volume element in spherical coordinates}

The transformation is as such
\[
  S \begin{pmatrix} \rho \\ \theta \\ \phi \end{pmatrix}  = \begin{pmatrix} x \\ y \\ z \end{pmatrix}  = \begin{pmatrix} 
  \rho\ cos \theta\ sin \phi \\
  \rho\ sin \theta\ sin \phi \\
  \rho\ cos  \phi 
\end{pmatrix} 
\] 

The derivative is 
\[
  \left[ D S \right]_{}  = \begin{bmatrix} 
     cos \theta\ sin \phi & -\rho\ sin \theta\ sin \phi & \rho\ cos \theta\ cos \phi \\  
     sin \theta\ sin \phi & \rho\ cos \theta\ sin \phi & \rho\ sin \theta\ cos \phi \\  
     cos \phi  & 0 & - \rho\ sin \phi
  \end{bmatrix}
\] 

And 
\[
  det \left[ D S \right]_{}  = \rho ^2 sin \phi
\] 

\section{Change of variables theorem}

\subsection{Recall U-SUBS!!!}

Recall from single variable that we can integrate by substitution via
\begin{align*}
   u &= F(x) \\
   du &= F'(x) dx
\end{align*}

What we really were doing was
\begin{enumerate}
   \item writing down a non-linear coordinate transformation
   \item finding the new length element
\end{enumerate}

Now we generalize into higher dimensions. Instead of calling it u-subs, we call it the \textbf{change of variables theorem}

\subsection{Change of variables theorem}

\begin{framed}
   If
   \[
     \underline{u} = F( \underline{x})
   \] 
   Then
   \[
     d \underline{u} = \left| Det \left[ D F \right]_{}  \right| d \underline{x}
   \] 

   Expressed in words, in arbitrary dimensions, the n-volume element in a new coordinate system is related to the volume element in the original coordinate system by the absolute value of the determinant of the transformation 
\end{framed}

This is a good point in the semester to look back and have our little minds blown by how far we've come. We can relate volume elements before and after non-linear transformations using little more than determinants and derivatives.  \\

In old school calculus textbooks that do not formally introduce the derivative as a linear transformation, this theorem would be a lot more verbose (involving a bunch of partial derivatives

\subsection{Examples with coordinate systems}

Consider an inverted cone (with a flat, circular base) symmetrical about the z-axis, with height $h$ , base radius $R$, with its tip at the origin. How would we describe this shape? Which coordinate system would we use? \\

\textbf{Spherical coordinates}
We can observe that
\begin{itemize}
   \item $0 \leq \theta \leq 2 \pi$ 
   \item $0 \leq \phi \leq arctan \left(  \frac{R}{h} \right) $ 
   \item $ 0 \leq \rho \leq ???????????$
\end{itemize}

\textbf{Cylindrical coordinates?}
\begin{itemize}
   \item $ 0 \leq z \leq h$ 
   \item $ 0 \leq \theta \leq 2 \pi$ 
   \item $ 0 \leq r \leq z \frac{R}{h}$ 
\end{itemize}

\subsection{Example, 9 Nov}

For a cube centered at origin with length $L$, with density $f = \frac{1}{ \left( x^2 + y^2 + z^2  \right)^{\alpha}}$ for some constant $\alpha > 0$, does the mass converge?

\[
   M = \int dM = \int_{- \frac{L}{2}}^{ \frac{L}{2}}   \int_{- \frac{L}{2}}^{ \frac{L}{2}}\int_{- \frac{L}{2}}^{ \frac{L}{2}} \left( x^2 + y^2 + z^2 \right)^{- \alpha} dx dy dz = ?????????????????
\] 

Trying spherical coordinates

\begin{align*}
   M &= \int dM \\ 
     &= \iiint \rho^{- 2 \alpha} \rho^2 sin \phi d \rho\ d \phi\ d \theta
\end{align*}

To determine if mass is finite, we only need to care about the asymptotic behavior, i.e. whether the mass of the infinitesimal sphere near the origin is finite

Hence, we can define the limits of integration as such for some $ \epsilon$
\begin{align*}
   M &= \int dM \\ 
     &= \int_{ \theta = 0}^{2\pi} \int_{ \phi = 0}^{\pi} \int_{\rho = 0}^{\epsilon}     \rho^{- 2 \alpha} \rho^2 sin \phi d \rho\ d \phi\ d \thetan \\
     &= \left. \frac{4 \pi \rho^{2 - 2\alpha + 1}}{3 - 2 \alpha
        } \right|_{\rho = 0}^{\epsilon} 
\end{align*}

Where is the badness here? \\

There are a few actually
\begin{itemize}
   \item denominator cannot be zero, hence $ \alpha \neq \frac{3}{2}$ 
   \item mass cannot be negative, hence $\alpha < \frac{ 3}{2}$ 
   \item integrating \[
         \int x^n dx = \frac{x ^{n+1}}{n+1}
   \] only works for $n \neq -1$, hence  $\alpha \neq \frac{3}{2}$
\end{itemize}

\subsection{Another example in changing variables}

We want to evaluate 
\[
   \iint_{D} xy (x^2 + y^2) dx dy
\] 

Where
\[
  D: 1 \leq xy \leq 4, 1 \leq x^2 - y^2  \leq 3
\] 

Recall that for
\[
  \underline{u} = F( \underline{x})
\] 
We have
\[
  d \underline{u} = \left| det \left[ D F \right]_{}  \right| d \underline{x}
\] 

We change coordinates to hopefully make bounds of integration constant!

\textbf{Let}:
\[
  F \begin{pmatrix} x \\y \end{pmatrix}  = \begin{pmatrix} u \\v \end{pmatrix}  = \begin{pmatrix}  xy \\ x^2 - y^2 \end{pmatrix} 
\] 

\[
  \left[ D F \right]_{} = \begin{bmatrix} 
     y & x \\ 2x \\ -2y  
  \end{bmatrix}
\] 

\[
  Det \left[ D F \right]_{}  = - 2y^2 - 2x^2
\] 

Hence we get
\begin{align*}
   dudv &= \left| -2y^2 -2x^2 \right| dxdy  \\
        &= 2 (x^2 + y^2) dx dy \\
   dxdy &= \frac{dudv}{2(x^2+y^2)}
\end{align*}

The integral is thus
\begin{align*}
   \iint_{D} \frac{xy (x^2 + y^2) dudv}{2 (x^2 + y^2)}  &= \int_{v = 1}^{3}  \int_{u = 1}^{4}  \frac{u}{2} du dv \\
                                                        &= \left. v \right|_{1}^{3} \left. \frac{u^2}{4} \right|_{1}^{4} 
\end{align*}

\section{Addenda!}

\subsection{Surface area}

For the surface area of a sphere of radius $R$, the surface area element is given by
\[
  d \sigma = R^2 sin \phi d \phi d \theta
\] 

This can be thought of as the volume element, except $\rho $ is now fixed at $R$

We can verify by taking the integral
 \[
  \int_{ \theta = 0}^{ 2 \pi}  \int_{ \phi = 0}^{ \pi}  d \sigma = 4 \pi R^2  
\] 

\subsection{Gaussian}

In 1D, it is given by
\[
   \frac{1}{\\sqrt{{2 \pi}}} e^{- \frac{x^2}{2}}
\] 

In 2D
\[
   \frac{1}{2 \pi} e^{ -\frac{1}{2} \left( x^2 + y^2 \right) } 
\] 





















