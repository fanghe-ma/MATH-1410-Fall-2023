\chapter{Week 13}

\section{Green's Theorem}

\subsection{The Theorem}
\begin{framed}
   Let $D \in \mathbb{R}^2$ be a bounded domain with oriented boundary $ \partial D$. Then, for $ P, Q$ differentiable on all of $D$, 
   \[
      \int_{\partial D}^{ } Pdx + Qdy = \iint_{D} \frac{\partial Q}{\partial x} - \frac{\partial P}{\partial y} dA 
   \] 
\end{framed}

Green's Theorem relates the integral of a 1-form over the \textbf{boundary of a domain} in the plane to the integral of a certain derivative over the \textbf{domain interior} \\

\textbf{For some special 1-forms...}, the resulting integral is equal to the area of the bounded domain
\begin{itemize}
   \item $\alpha = xdy $
      \[
         \int_{\gamma}^{}   xdy= \iint_{D} \frac{\partial }{\partial x}x dA = A
      \] 
\end{itemize}

\subsection{Orientation and Boundaries}

Orientation on a planar domain can either be
\begin{itemize}
   \item positive: counterclockwise (by default)
   \item negative: clockwise
\end{itemize}

The default is to assume a positive orientation on a planar domain. Such an orientation induces a counterclockwise rotation on the boundary. \\

A simply connected domain has one boundary component. \\

A multiply connected domain has multiple boundary components, consistent with the notion of a infinitesimal counterclockwise rotation.

\subsection{Differentiability, orientation \& Green's Theorem}

Recall that in order to apply Green's Theorem over $D$, the 1-form must be differentiable on \textbf{all of $D$}

Consider a 1-form
\[
  \alpha = \frac{-y}{x^2 + y^2} dx + \frac{x}{x^2 + y^2}dy
\] 

Suppose we want to integrate the 1-form on $\mathbb{R}^2$ over the counterclockwise unit circle $u$, we can compute the integral using u-sub
\begin{align*}
   \int_{u}^{}  \frac{-y}{x^2 + y^2} dx + \frac{x}{x^2 + y^2}dy &= 
   \int_{t = 0 }^{2\pi}  \frac{-sint (-sint)}{sin^2 t + cos^2 t} + \frac{cost (cost)}{sin^2 t + cos^2 t} dt \\
                                                                &= 2\pi
\end{align*}

However, if we try to use Green's Theorem, the same integral becomes

\[
   \iint_{D} \frac{\partial }{\partial x} \frac{x}{x^2 + y^2} - \frac{\partial }{\partial y} \frac{-y}{x^2 + y^2} dA = \iint_{D} \frac{(x^2 + y^2) - 2x^2 + (x^2 + y^2) - 2y^2}{(x^2+ y^2)^2 } = 0 \neq 2\pi
\] 

This is because the 1-form is not differential at the origin \\

Now, if we want to integrate the 1-form over some arbitrarily complicated, singly connected domain $D_0$ that includes the origin, we can compute the integral by instead computing the integral over a \textbf{multiply connected} domain $D_1$ with the unit circle cut out \\

Note that $D_1$ has 2 boundaries, the outer boundary identical to $D_0$, and an inner \textbf{clockwise boundary}  $u$, which is the unit circle. \\

Green's Theorem now applies to $D_1$, and the integral, as computed previously, is
\[
   \iint_{D_1} \frac{\partial }{\partial x} \frac{x}{ x^2 + y^2} - \frac{\partial }{\partial y} \frac{-y}{ x^2  + y^2} dA = 0
\] 

Let the outer boundary be $\gamma$, inner boundary be $u$
 \[
  \int_{\gamma - u}^{ }  \alpha = \int_{\gamma}^{}  \alpha - \int_{u}^{} \gamma = 0
\] 

\section{Gradient, Curl \& Divergence}
This section combines differentiation, 1-forms and what we learned previously about the gradient

\subsection{Curl and divergence densities}

For a vector field 
\[
  \vec{F} = F_x  \underline{i} + F_y \underline{j}
\] 

We can write down two new versions of Green's Theorem, for circulation and flux respectively

\textbf{Circulation}
\[
  \int_{\partial D}^{}  F_x dx + F_y dy = \iint_D \frac{\partial F_y}{\partial x} - \frac{\partial F_x}{\partial y}  dA
\] 

\textbf{Flux}
\[
  \int_{\partial D}^{}  F_x dy - F_y dx = \iint_D \frac{\partial F_x}{\partial x} + \frac{\partial F_y}{\partial y}  dA
\] 

We can think of the integrands of the double integrals as some form of \textbf{densities}. In fact, they are two different types of \textbf{derivatives} of the vector fields

\textbf{Curl}
\begin{framed}
   \textbf{Curl} indicates, at each point in a vector field, the infinitesimal counterclockwise spin exhibited by the vector field. It is a limiting intensity of circulation \\
 
   Density is given by

   \[
     \frac{\partial F_y}{\partial x} - \frac{\partial F_x}{\partial y}
   \] 

   It is denoted as 
   \[
     \nabla \times \vec{F}
   \] 

\end{framed}

\textbf{Divergence}
\begin{framed}
   \textbf{Divergence} indicates, at each point in a vector field, the infinitesimal expansion or contraction exhibited by the vector field. It is a limiting intensity of flux \\

   Density is given by

   \[
     \frac{\partial F_x}{\partial x} + \frac{\partial F_y}{\partial y}
   \] 

   It is denoted as 
   \[
     \nabla \cdot \vec{F}
   \] 
\end{framed}

\subsection{Green Theorem through the perspective of curl and divergence}

Given a vector field
\[
  \vec{F} = F_x \underline{i} + F_y \underline{j}
\] 

For $D \in \mathbb{R}^2$ a domain with oriented boundary $ \partial D$

%\begin{align*}
%   \int_{\partial D}^{}  F_x dx + F_y dy  = \iint_{D} \left( 
%      \frac{\partial F_y}{\partial x} - \frac{\partial F_x}{\partial y    \right)  dA &= \iint_D \left( \nabla \times \vec{F} \right) dA \\
%      \int_{\partial D}^{} - F_y dx + F_x dy  = \iint_{D} \left( 
%      \frac{\partial F_x}{\partial x} + \frac{\partial F_y}{\partial y    \right)  dA &= \iint_D \left( \nabla \cdot \vec{F} \right) dA
%\end{align*}

\subsection{Divergence in 3-D}
\begin{framed}
   The divergence of a 3-D vector field is a scalar field. \\

   The divergence at a point indicates the local expansion or contraction of the \textbf{volume element}

   \[
     \nabla \cdot \vec{F} = \frac{\partial F_x}{\partial x} + \frac{\partial F_y}{\partial y} + \frac{\partial F_z}{\partial z}
   \] 
\end{framed}

\subsection{Curl in 3-D}
\begin{framed}
   The curl of a 3-D vector field is another vector field. \\

   The formula for curl has as its components the three circulation densities in each plane

   \[
     \nabla \times \vec{F} = Det \begin{bmatrix} 
        i & \frac{\partial }{\partial x} & F_x   \\
        j & \frac{\partial }{\partial y} & F_y   \\
        k & \frac{\partial }{\partial z} & F_z   \\
     \end{bmatrix}
   \] 

   \[
     \nabla \times \vec{F} = 
     \left( \frac{\partial F_z}{\partial y} - \frac{\partial F_y}{\partial z}  \right)  \underline{i} + 
     \left( \frac{\partial F_x}{\partial z} - \frac{\partial F_z}{\partial x}  \right)  \underline{j} + 
     \left( \frac{\partial F_y}{\partial x} - \frac{\partial F_x}{\partial y}  \right)  \underline{k} 
   \] 
  
\end{framed}


\section{Differential Forms in 3-D}

\subsection{Euclidean Forms}

\subsubsection{1-forms}

In 3-D, we have the basis 1-forms which give \textbf{oriented projected length} along the basis axes
\[
  dx, dy, dz
\] 

From the basis 1-forms, a \textbf{linear} 1-form is of the following form, and returns \textbf{oriented projected length} along some direction
\[
  \alpha = a\ dx + b\ dy + c\ dz
\] 

A 1-form field is one of the form
\[
  \alpha = f_1 dx + f_2 dy + f_3 dz
\] 

\subsubsection{2-forms}

In general, a k-form takes $k$ ordered vectors and returns a scalar

In 3-D, we have basis 2-forms which give \textbf{oriented projected area} 
\[
  dx \wedge dy, dy \wedge dz, dz \wedge dx
\] 

Each basis 2-form takes 2 vectors and returns some scalar, where 
\begin{itemize}
   \item $dx$ extracts the $x$ component
   \item $dy$ extracts the $y$ component
   \item $dz$ extracts the $z$ component
\end{itemize}

The 2-forms are thus
\begin{align*}
   \left( dx \wedge dy \right) ( \underline{u}, \underline{v}) &= Det \begin{bmatrix} 
      u_x & v_x \\ u_y & v_y  
   \end{bmatrix} \\
   \left( dy \wedge dz \right) ( \underline{u}, \underline{v}) &= Det \begin{bmatrix} 
      u_y & v_y \\ u_z & v_z  
   \end{bmatrix} \\
   \left( dz \wedge dx \right) ( \underline{u}, \underline{v}) &= Det \begin{bmatrix} 
      u_z & v_z \\ u_x & v_x  
   \end{bmatrix}
\end{align*}

A linear 2-form is one of the form
\[
  \beta = a\ dx \wedge dy + b\ dy \wedge dx + c\ dz \wedge dx
\] 

A 2-form field is one of the form
\[
  \beta = f_1 dx \wedge dy + f_2 dy \wedge dz + f_3 dz \wedge dx
\] 

\subsubection{3-form}

The basis 3-form is 
\[
  dx \wedge dy \wedge dz
\] 

The basis 3-form gives the oriented volume when given an ordered triplet of vectors $ \left( \underline{u}, \underline{v}, \underline{w} \right) $ in $ \mathbb{R}^3$

\[
   \left( dx \wedge dy \wedge dz \right) \left(  \underline{u},  \underline{v}, \underline{w} \right)  = Det \begin{bmatrix} 
      u_x & v_x & w_x \\  
      u_y & v_y & w_y \\  
      u_z & v_z & w_z
   \end{bmatrix}
\] 


\subsection{The Wedge}

The wedge product is a symbolic representation of stacking vectors into a determinant. 

The rules are
\begin{itemize}
   \item $ dx_i \wedge dx_j = - dx_j \wedge dx_i$, 
   \item $ dx_i \wedge dx_i = 0$
\end{itemize}

\subsection{Differentiation of form fields}

Recall that the derivative of a scalar field $f$ (0-form field) is a gradient 1-form

\[
  df = \frac{\partial f}{\partial x} dx + \frac{\partial f}{\partial y}dy + \frac{\partial f}{\partial z}dz
\] 

Differentiating a 1-form field, we have
\[
  d(f dx_i) = df \wedge dx_i
\] 

For example,
\begin{align*}
   d(f\ dx) &= df \wedge dx \\
            &= \frac{\partial f}{\partial y} dy \wedge dx + \frac{\partial f}{\partial z} dz \wedge dx
\end{align*}

In general, taking the derivative of a $k$-form field gives a $k+1$ form field

\subsubsection{Flux}
\begin{framed}
\[
   \phi_{\vec{F}} ( \underline{u}, \underline{v}) = Det \left[ \vec{F} | \underline{u} | \underline{v} \right]
\] 

\[
   \phi_{\vec{F} } = F_x dy \wedge dz + F_y dz \wedge dx + F_z dx \wedge dy
\] 
\end{framed}

\subsubsection{Curl}
\begin{framed}
Computing the derivative of 
\[
  \alpha = f_1 dx + f_2 dy + f_3 dz
\] 

\[
  d \alpha = df_1 \wedge dx + df_2 \wedge dy + df_3 \wedge dz
\] 

Where
\[
  df_i = \frac{\partial f_i}{\partial x }  dx + \frac{\partial f_i}{\partial y} dy + \frac{\partial f_i}{\partial z}dz
\] 

\begin{align*}
   d\alpha &= 
   \left( \frac{\partial f_1}{\partial y}dy  + \frac{\partial f_1}{\partial z} dz\right) \wedge dx + 
   \left( \frac{\partial f_2}{\partial x}dx  + \frac{\partial f_2}{\partial z} dz\right) \wedge dy + 
   \left( \frac{\partial f_3}{\partial x}dx  + \frac{\partial f_3}{\partial y} dy\right) \wedge dz  \\
           &= 
           \left( \frac{\partial f_3}{\partial y} - \frac{\partial f_2}{\partial z} \right) dy \wedge dz
           \left( \frac{\partial f_1}{\partial z} - \frac{\partial f_3}{\partial x} \right) dz \wedge dx
           \left( \frac{\partial f_2}{\partial x} - \frac{\partial f_1}{\partial y} \right) dx \wedge dy
\end{align*}

Hence 
\[
   d \alpha_{ \vec{F}} = \phi_{\nabla \times \vec{F}}
\] 
\end{framed}


\subsubsection{Divergence}
\begin{framed}
Finding the derivative of
\[
  \beta = f_1 dy \wedge dx + f_2 dz \wedge dx + f_3 dx \wedge dy
\] 

\[
  d \beta = \left( \frac{\partial f_1}{\partial x} + \frac{\partial f_2}{\partial y} + \frac{\partial f_3}{\partial z} \right) dx \wedge dy \wedge dz
\] 
\end{framed}


