\chapter{Week 14}

This week is the end. \\

Bulk of this week will be spent understanding
\[
  \int_{\partial D}^{} \alpha = \int_{D}^{d\alpha}   
\] 

Where 
\begin{itemize}
   \item $\alpha$ is any $k$-form field
   \item $D$ is some domain of dimension $k + 1$
\end{itemize}

This is \textbf{the} Stoke's Theorem, and this week's content covers a specific case of Stoke's Theorem. \\

This is the same idea as 
\begin{itemize}
   \item Fundamental theorem of integral calculus
   \item Independence of path (2-D)
   \item Green's Theorem (2-D)
   \item Gauss' Theorem (3-D)
   \item Classical Stoke's Theorem (3-D)
\end{itemize}

\section{Relooking Green's Theorem}

Recall from last week that Green's Theorem can be written in several different forms. 

\subsection{Work / Circulation}

Given a vector field \[
  \vec{F} = F_x \underline{i} + F_y \underline{j}
\] 

The work done by the vector field along the boundary can be computed by taking the double integral of some density over the internal area. \\

The density is related to the vector field. It is known as the curl. \\

Curl is a concept that only makes sense in 3-D. In the case of 2-D, we are looking at the component of curl about the $ \underline{k}$  axis

\[
   \int_{\partial D}^{}  F_x dx + F_y dy = \iint_{D} \left( \nabla \times \vec{F} \right) \cdot \vec{k} dA  = \iint_D \frac{\partial F_y}{\partial x} - \frac{\partial F_x}{\partial y} dA
\] 

\subsection{Flux}

The flux of the vector field over the boundary can be computed by some combination of the partial derivatives over the internal area. \\

The integrand is known as the divergence. \\

\[
  \int_{\partial D}^{}  F_x dy - F_y dx  = \iint_D \left( \nabla \cdot \vec{F} \right) dA = \iint_D \frac{\partial F_x}{\partial x} + \frac{\partial F_y}{\partial y} dA
\] 

\subsection{Going back to the general equation}

Note that these two are both in the form
\[
  \int_{\partial D}^{}  \alpha = \int_{D}^{} d\alpha  
\] 

What is $D$ and $d\alpha$??? \\

In the 2-D cases mentioned above, 
\begin{itemize}
   \item  $D$ : 2-D domain
   \item $d\alpha$ : 2-form in $\mathbb{R}^2$
\end{itemize}

Since in 2-D, there is only 1 2-form
\[
  d\alpha = f(x, y) dx \wedge dy
\] 

Hence, we \textbf{define by fiat} that in $\mathbb{R}^2$
\[
  \int_{D}^{} f dx \wedge dy = \iint_D f dA 
\] 

\subsection{Lifting up by 1 dimension}

Consider the following integral. What is it?
\[
  \int_{S}^{} \beta 
\] 

Where 
\begin{itemize}
   \item $\beta$ is some 2-form field in 3-D
   \item $S$ is some oriented surface in $\mathbb{R}^3$
\end{itemize}

Recall that given $\alpha$, a 1-form field, we integrate it over a path, since 1-form operates on a single vector each. \\

Now, we consider surface in $\mathbb{R}^3$ \\

Let $S$ be the parameterized surface in $\mathbb{R}^3$. Given an implicit surface 
\[
  z = x^2 + y^2
\] 

\[
  S(s, t) = \begin{pmatrix} s \\ t \\ s^2 + t^2 \end{pmatrix}, s^2 + t^2 \leq R^2
\] 

Think of $S$ as mapping the 2-D plane of $s$ and $t$ (within the specified domain)  to a surface in 3-D \\

Consider the integral

\[
  \int_{S}^{} dx \wedge dy + dy \wedge dz 
\] 

Note that 
\begin{itemize}
   \item $dx \wedge dy$ measures oriented projected area in the $x, y$ plane, $= \pi R^2$
   \item $dy \wedge dz$ measures oriented projected area in the $y, z$ plane  $= 0$, because the parabola has two oriented projected areas, oriented in opposite directions
\end{itemize}

The integral is thus
\begin{align*}
   &\int_{S}^{} dx \wedge dy + dy \wedge dz  \\
   =& \pi R^2 + 0 \\
   =& \pi R^2
\end{align*}

Now we return to the theorem and compute the integral again \\

Taking the derivative of $S$, 
\[
   S(s, t) = \begin{pmatrix} s \\ t \\ s^2 + t^2 \end{pmatrix} , 
   \left[ D S \right]_{} = \begin{bmatrix} 
      1 & 0 \\ 0 & 1 \\ 2s & 2t  
   \end{bmatrix}
\] 

We now have another way to compute the integral over the surface $S$ 
\begin{align*}
   &\int_{S}^{} dx \wedge dy + dy \wedge dz \\
   =& \iint_{s'} (dx \wedge dy + dy \wedge dz) \left[ D S \right]_{} ds dt  \text{ where $s'$ is the circular domain in $s, t$}\\ 
   =& \iint_{s'}(dx \wedge dy) \begin{bmatrix} 
      1 & 0 \\ 0 & 1 \\ 2s & 2t  
   \end{bmatrix} + (dy \wedge dz) \begin{bmatrix} 
      1 & 0 \\ 0 & 1 \\ 2s & 2t  
   \end{bmatrix} ds dt \\
      =& \iint_{s'} Det \begin{bmatrix} 
         1 & 0 \\ 0 & 1  
      \end{bmatrix} + Det \begin{bmatrix} 
         0 & 1 \\ 2s & 2t  
      \end{bmatrix} ds dt \\
         =& \iint_{s'} 1 - 2s ds dt \\
         =& \iint_{s'} dsdt - \iint 2s ds dt \\
         =& \pi R^2 - 0 \text{ since $2s$ is an odd function integrated over a symmetric domain}
\end{align*}

Generalizing

\[
   \int_{S}^{} \beta = \iint_{S'} \beta_{S(s, t)} \left[ D S \right]_{} ds dt
   = \iint_{S'}  \beta_{S(s, t)} \frac{\partial S}{\partial s} \frac{\partial S}{\partial t} ds dt
\] 

The question is why???? \\

One good applications is flux, which is given by the flux 2-form

\[
   \Phi_{\vec{F}} = F_x dy \wedge dx + F_y dz \wedge dx + F_z dx \wedge dy
\] 

For some curve $S$, 

\[
   \int_{S}^{} \Phi_{\vec{F}} = \text{Flux of $\vec{F}$ across $S$} 
\] 

In the classical physics style, flux is calculated by taking some infinitesimal surface area element $d\sigma$, with an associated unit normal vector  $\hat{ \underline{n}}$
\begin{align*}
   \text{Flux} &= \iint \left( \vec{F} \cdot \hat{ \underline{n}} \right) d \sigma  \\
               &= \int_{S}^{} \Phi_{\vec{F}} 
\end{align*}

\section{Green's Theorem in 3D}
\begin{align*}
   \text{"Gauss" } :& \text{ "Div"} \\
   \text{"Stokes" } :& \text{ "Circulation?"} \\
\end{align*}

\subsection{Gauss' Theorem}

Guass: What is the flux of $\vec{F}$ across the boundary of some 3-D domain?  \\

The boundary of $D$, $\partial D$ is some \textbf{closed 2-D surface}

\begin{framed}
   \textbf{Theorem:} The flux over the boundary is equal to the integral of the derivative of the flux 2-form over the interior of the domain.

   \begin{align*}
      \int_{\partial D}^{} \Phi_{\vec{F}} &= \int_{D}^{} d \Phi_{\vec{F}}   \\
                                          &= \iiint_{D} \left( \nabla \cdot \vec{F} \right) dV
   \end{align*}

   The classical physics version of the theorem is stated as 
   \[
      \iint_{\partial D} \left( \vec{F} \cdot \hat{ \underline{n}} \right)  d\sigma = \iiint{D} \left( \nabla \cdot \vec{F} \right) dV
   \] 
\end{framed}

\subsection{Stokes' Theorem}

What is the circulation of a vector field $\vec{F} $ in $\mathbb{R}^3$ over some oriented 2-D domain (some surface in 3-D) with a boundary $\partial D$ ? \\  


Recall that the derivative of any 1-form field gives a 2-form field, which is itself the flux 2-form of some function 

\begin{framed}
   \textbf{Theorem:} The circulation over the boundary is equal to the integral of the flux 2-form of the curl of $\vec{F}$ over the interior of the domain.

   \begin{align*}
      \int_{\partial D}^{} \alpha_{\vec{F}} = \int_{D}^{} d \alpha_{\vec{F}}   
   \end{align*}

   \[
     \int_{\partial D}^{} \vec{F} \cdot d \underline{x}  = \iint_D \left( \nabla \times \vec{F} \right) \cdot \hat{ \underline{n}} d\sigma
   \] 

   The circulation of $F$ on $\partial D$ is equal to the flux of $\nabla \times \vec{F}$ across $D$

\end{framed}

\subsection{Dropping back down to Greens in 2-D}

Recall that 
\begin{align*}
   \text{circulation on $\partial D$} &= \text{Flux of $\nabla \times \vec{F} \cdot \hat{k}$} \rightarrow \text{Stokes} \\
   \text{flux of $\vec{F}$ across $\partial D$} &= \iint_D Div(\vec{F}) \rightarrow \text{Gauss}
\end{align*}


\subsection{An example}

Given
\[
  \vec{F} = \left( x^2 + y^2 \right) \hat{i} + 2e^z \hat{j} + (x^2 + z^2) \hat{k}
\] 

Compute the flux of $\vec{F}$ across a cube $0 \leq x \leq 1$, $0 \leq y \leq 1$,  $0 \leq z \leq 1$

By Gauss,

\begin{align*}
   \text{Flux} &= \int_{0}^{1} \int_{0}^{1} \int_{0}^{1}  \nabla \cdot \vec{F} dx dy dz \\
               &= \int_{0}^{1} \int_{0}^{1}    \int_{0}^{1}  2x + 2z dx dy dz \\
               &= 1 + 1  \\
               & 2
\end{align*}


\section{Thursday Lecture}


Recall that by Stoke's Theorem, in $\mathbb{R}^n$

\[
  \int_{\partial D}^{} \alpha = \int_{D}^{}  d \alpha 
\] 

Where $\partial D$ is some boundary and $D$ is in $k$ dimensions, and $\alpha$ is a k-form

\begin{center}
   \begin{tabular}{c c c c} 
      n & k & &\\
      2 & 1 & Green's&\\
      3 & 1 & Stokes&\\
      3 & 2 & Gauss&\\
      1 & 0 & FTIC &($f(b) - f(a) = \int_{{a, b}}^{} f - \int_{{a,b}}^{} df $)\\
      n & 0 & IoP& \\
   \end{tabular}
\end{center}


\subsection{Practice problem 1}

Given a vector field 
\[
  \vec{F} = y^2 \hat{i} - x^2 \hat{j} + z \hat{k}
\] 

What is the flux of $\vec{F}$? \\

When we see \textbf{flux}, we ask ourselves, is the question asking for the \textbf{flux of $f$} or \textbf{flux of the curl}? \\

In this case, they are asking for the flux of $f$, so we know that we \textbf{don't need Stokes} \\

Can we use Gauss? Gauss calculates flux but it applies to a closed surface. In this case, we want the flux over an open top hemisphere. Hence no Gauss \\

What we can do is a direct integration 
\[
   \Phi_{\vec{F}} = y^2 dy \wedge dz - x^2 dz \wedge dx + 2 dx \wedge dy
\] 

This is doable, but it is not obvious / easy  \\


\[
  S(s, t) = \begin{pmatrix} s \\ t \\ \sqrt{R^2 - s^2 - t^2} \end{pmatrix} 
\] 

\[
   \left[ D S \right]_{}  = \begin{bmatrix} 
      1 & 0 \\
      0 & 1 \\
      ???? & ???
   \end{bmatrix}
\] 

\textbf{THINK MARK THINK!!!}

We go back to the flux two form
\[
   \Phi_{\vec{F}} = y^2 dy \wedge dz - x^2 dz \wedge dx + 2 dx \wedge dy
\] 

Because of the 2-forms, we are working with the projected area onto each of the 3 planes \\

Consider the $y-z$ plane, the net flux is zero \textbf{by symmetry}
\[
   \int_{}^{}  y^2 dy\wedge dz = 0
\] 

Likewise in the $x-z$ plane, the net flux is also zero \textbf{by symmetry}
\[
   \int_{}^{}  -x^2 dx\wedge dz = 0
\] 

The only non-vanishing net flux is given by
\[
   \int_{}^{}  2 dx\wedge dy = 2 \text{, the oriented projected area in the $x, y $ plane} 
\] 

\textbf{NOTE}
In general, \[
  \int_{}^{} f(x, y) dx \wedge dy  = 0
\] 
Or
\[
  \int_{}^{} f(x_i, x_j) dx_i \wedge dx_j = 0 
\] 

\textbf{WE ARE NOT DONE}

We will try using Gauss again, because we are stubborn \\

We now make the domain a solid hemisphere, such that its boundary has two pieces \\

By Gauss, the net flux is 
\[
   \int_{\text{spherical boundary}}^{} \Phi - \int_{\text{circular boundary at base}}^{}  \Phi  = \iiint_{D} \nabla \cdot \vec{F} dV 
\] 

We then show that the divergence is 0

\[
   \text{Verify!}
\] 

Hence 
\[
   \int_{\text{spherical boundary}}^{} \Phi - \int_{\text{circular boundary at base}}^{} \Phi   = \iiint_{D} \nabla \cdot \vec{F} dV  = 0
\] 

Now we find
\[
   \int_{\text{circular boundary at base}}^{} \Phi = \int_{}^{} 2 dx \wedge dy 
\] 

Which is the same computation we did above

\subsection{Another example}

Given a vector field
\[
   \vec{F} = (y + x sinx^2) \hat{i} + \left( x^2 + e^{y^2 -5y} \right)\hat{j} + (x^2 + y^2) \hat{k} 
\] 

Find the flux of the curl of $\vec{F}$ over the surface defined by \[
  z = cos^3 \left( \frac{\pi}{2} \left( x^2 + y^2 \right)  \right), x^2 + y^2 \leq 1
\] 

By Stokes, the integral over the boundary of the surface of the work 1-form of $f$ is equal to the integral over the interior of $D$ of the derivative of the work 1-form

\[
   \int_{\partial S}^{}  \alpha_{\vec{F}} = \int_{S}^{} d \alpha_{\vec{F}} = \int_{S}^{} \Phi_{\nabla \times \vec{F}}
\] 

The right hand side is what the question asked for, the left hand side is easier to compute \\

Along the boundary of $S$,
\[
  z = cos^3 \left(  \frac{\pi}{2} (x^2 + y^2) \right)  = cos^3 \frac{\pi}{2} = 0
\] 

\textbf{The strategy}

We compute the circulation over the boundary, by computing
\begin{align*}
   \int_{\partial S}^{}  \alpha_{\vec{F}}  \int_{t = 0}^{2 \pi}  \left( sint + cost sint cos^2 t \right) \left( -sint \right) + \left( ???? \right) cost dt
\end{align*}

Where the boundary is parameterized as 
\[
  \gamma(t) = \begin{pmatrix} cos t \\ sint  \end{pmatrix} 
\] 

This is not going to work....

\textbf{We try again}

We use the independence of surface, as long as the boundary remains the same. \\

We use the surface \[
   z = 0, x^2 + y^2 \leq 1
\] 

We want to calculate
\[
   \int_{}^{} \Phi_{\nabla \times \vec{F}}  
\] 

We write down the work 1-form
\[
   \alpha_{\vec{F}} = \left( y^2 +xsinx^2 \right) dx + \left( x^2 + e^{y^2 - 5y} \right) dy + \left( x^2 + y^2 \right) dz
\] 

The derivative of work 1-form is
\begin{align*}
   d\alpha_{\vec{F}} &= dy \wedge dx + 2x dx \wedge dy + 2x dx \wedge dz + 2y dy \wedge dz \\
                     &= (2x -1) dx \wedge dy + 2x dx \wedge dz + 2y dy \wedge dz
\end{align*}

The projected area of $2x$ in the $x-z$ plane is 0 \\

The projected area of $2y$ in the $y-z$ plane is 0 \\

Hence
\[
   \int_{\text{disc}}^{}  d \alpha_{\vec{F}} = \iint 2x - 1 dx \wedge dy = 2 \iint x dA - \iint dA = -\pi 
\] 

The derivative of the work 1-form is equal to the flux of the curl!!!

\[
   d\alpha_{\vec{F}} = \Phi_{\nabla \times \vec{F}}
\] 

\subsection{Another another example}

For an object submerged in a fluid of constant uniform density $\rho$,

The pressure exerted by the fluid is  \textbf{orthogonal} to the surface of the object \\

The net force in the $x$ and $y$ directions are 0 \\

Buoyancy is related to the oriented projected area in the $z$ direction, i.e. in the $x-y$ plane \\

Hence, buoyancy is a 2-form! Taking into account that buoyancy is also a function of depth, given by the $z$ coordinate, we get

\[
  \beta = \rho z dx \wedge dy
\] 

The net buoyancy force is 
\[
   \int_{\text{skin}}^{}  \beta = \int_{\partial D}^{}  \beta 
\] 

By Gauss Theorem

\begin{align*}
   &\int_{\partial D}^{}  \beta  \\
   =& \int_{D}^{} d\beta \\
   =& \iiint_{D} \rho dz\wedge dx \wedge dy   \\
   =& \iiint_{D} \rho dV \\
   =& \rho \times \text{Volume}  \\
   =& mass
\end{align*} 

