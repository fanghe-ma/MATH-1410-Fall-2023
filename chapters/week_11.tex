\chapter{Week 11}

\section{Polar \& Cylindrical coordinates}

\subsection{Polar Coordinates}

\begin{framed}
   The area element in polar coordinates is given by
   \[
     dA = r\ dr\ d \theta
   \] 

   The infinitesimal area element is given by the product of
   \begin{itemize}
      \item $dr$ in the radial direction
      \item $r d \theta$ in the angular direction
   \end{itemize}

   The transformation is 
   \[
      \begin{array}{c c}
         x = r cos \theta & r = \sqrt{x^2 + y^2} \\
         y = r sin \theta & \theta = arctan( \frac{y}{x})
      \end{array}
   \] 
\end{framed}

\subsection{Cylindrical Coordinates}
\begin{framed}
   The volume element in cylindrical coordinates is given by
   \[
     dV = r\ dr\ d \theta dz
   \] 

   The infinitesimal volume element is given by the product of
   \begin{itemize}
      \item $dr$ in the radial direction
      \item $r d \theta$ in the angular direction
      \item $z$ in the $z$ axis
   \end{itemize}

   The transformation is 
   \[
      \begin{array}{c c}
         x = r cos \theta & r = \sqrt{x^2 + y^2} \\
         y = r sin \theta & \theta = arctan( \frac{y}{x})
      \end{array}
   \] 
\end{framed}

\section{Spherical coordinates}
\begin{framed}

   The volume element is given by
   \[
     dV = \rho^2 sin\ \phi d \rho\ d \theta\ d \phi
   \]  


   Spherical coordinates are defined by 
   \begin{itemize}
      \item radius $\rho$ 
      \item inclination $\phi$ 
      \item azimuth $\theta$ 
   \end{itemize}

   Standard coordinates and spherical coordinates can be translated by
   \begin{align*}
      x &= \rho\ cos\ \theta\ sin\ \phi \\
      y &= \rho\ sin\ \theta\ sin\ \phi \\
      z &= \rho\ cos\ \phi
   \end{align*}

   Where 
   \begin{align*}
      \rho & \geq 0 \\
      0  & \leq \theta  \leq 2 \pi \\
      0  & \leq \phi  \leq \pi \\
   \end{align*}

   To convert standard coordinates to polar coordinates
   \begin{align*}
      \rho &= \sqrt{x^2 + y^2 + z^2}  = \sqrt{r^2 + z^2}\\
      \theta &= arctan( \frac{y}{x}) \\
      \phi &= arctan \left( \frac{r}{z} \right) = arccos \left( \frac{z}{\rho } \right) 
   \end{align*}
\end{framed}

\section{Change in Variables Theorem}

\subsection{Linear transformations and the volume element}

Recall from earlier in the semester that
\begin{itemize}
   \item The determinant of a square matrix computes a volume
   \item Square matrices encode linear transformation  
\end{itemize}

For a linear transformation $A$ that takes $ \underline{x}$ coordinates to $ \underline{u} $ coordinates, 
\[
  \underline{u} = A \underline{x}
\] 

The volume element is transformed as 
\[
  d \underline{u} = \left| Det \left( A \right)  \right| d \underline{x}
\] 


\subsection{Non-linear transformations and the volume element}
Given a change of coordinates $F: \mathbb{R}^n \rightarrow \mathbb{R}^n$ of the form 
\[
  \underline{ u} = F \left(  \underline{x} \right)
\] 

The volume element is transformed as
\[
   d \underline{u} = \left| Det \left[ D f \right]_{}  \right| d \underline{x}
\]  

\subsection{Change of variables theorem}

Recall that in single variable integral calculus, u-sub is given by
\[
   \int_{u_0}^{u_1}  h(u) du = \int_{x_0}^{x_1}  h(u(x)) \frac{du}{dx} dx  
\] 

\begin{framed}
   \textbf{Theorem}: Given a change of coordinates $F: \mathbb{R}^n \rightarrow \mathbb{R}^n$ of the form 
   \[
     \underline{u} = F \left( \underline{x} \right) 
   \] 

   The integral of $h$ over a region $F(R)$ is converted as
    \[
       \int_{F(R)} h \left( \underline{u} \right) d \underline{u} = \int_R h \left( F \left( \underline{x} \right)  \right) \left| Det \left[ D F \right]_{}  \right| d \underline{x}
   \] 
\end{framed}

\section{Surface Integrals}

\subsection{Parameterized surfaces}

In 3-D, a parameterized surface has two inputs and three outputs, i.e.
\[
  G \begin{pmatrix} s\\t \end{pmatrix}  = \begin{pmatrix} x \\ y \\ z \end{pmatrix}  = \begin{pmatrix} 
    x(s, t)  \\
    y(s, t)  \\
    z(s, t)  
  \end{pmatrix}
\] 

The derivative is a matrix with two columns, each of which gives a tangent vector.
\[
  \left[ D G \right]_{}  = \begin{bmatrix} 
     & \\
     \frac{\partial G}{\partial t} & \frac{\partial G}{\partial s} \\
                                   &
  \end{bmatrix}
\] 

The infinitesimal surface area element is given by the area of a parallelogram, bounded by $\frac{\partial G}{\partial t}$ and $\frac{\partial G}{\partial s}$
\[
   \text{area of parallelogram} = \left| \frac{\partial G}{\partial t} \times \frac{\partial G}{\partial s} \right| 
\] 

Hence,
\begin{framed}

   For a surface $S \in R^3$ parameterized by $ G : \mathbb{R}^2  \rightarrow \mathbb{R}^3 $
   \[
     G \begin{pmatrix} s \\t \end{pmatrix}  = \begin{pmatrix}  x \\ y \\z \end{pmatrix} 
   \] 
   The surface area $\sigma$ is given by
   \[
      \int_{S = G(R)} d \sigma = \iint_R \left| \frac{\partial G}{\partial s} \times \frac{\partial G}{\partial t} \right| ds\ dt
   \]  
  
\end{framed}


In spherical coordinates
\[
  d \sigma = \mathbb{R}^2 sin \phi d \theta\ d \phi
\]  


