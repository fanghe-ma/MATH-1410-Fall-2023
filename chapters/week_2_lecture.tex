\chapter{Week 2 lecture notes}

\section{Recap from previous week}
From last week
\begin{itemize}
 \item The n-dimensional euclidean space $\mathbb{R}^n$
 \item functions where \[
   f : \mathbb{R}^n \rightarrow \mathbb{R}^m
 \]  
\end{itemize}

\section{This week}

\begin{itemize}
   \item   enrich algebraic manipulation of vectors where $ \underline{v} = \begin{pmatrix} v_1 \\v_2 \\ v_3 \\\vdots \\ v_n \end{pmatrix} $ 
   \item preview of calculus with vectors
  
\end{itemize}


\section{The dot product}

\begin{framed}
   For two vectors $ \underline{v}$ and $ \underline{u}$, their dot product is
   \begin{align*}
      \underline{u} \cdot \underline{v} &= \sum_{k = 1}^{n} u_k v_k
   \end{align*}

   Note the similarity with finding the \emph{norm} of a vector
   \begin{align*}
      \lVert \underline{v} \rVert  &= \sqrt{v_1^2 + v_2^2 + v_3^2 + \hdots + v_n^2} \\
      \underline{v} \cdot \underline{v} &= \lVert \underline{v} \rVert^2
   \end{align*}

   Note some intuitive rules for the dot product, and their geometric meaning
   \begin{align*}
      \underline{u} \cdot ( \underline{v} + \underline{w}) &= \underline{u} \cdot \underline{v} + \underline{u} \cdot \underline{w} \\
      \underline{v} \cdot \underline{0} &= 0
   \end{align*}

   \textbf{IMPORTANT - Commit to memory!!} \\
   For vectors $ \underline{v}$ and $ \underline{u}$ separated by angle $ \theta$
   \begin{align*}
      \underline{u} \cdot \underline{v} &= \lVert \underline{u} \rVert  \lVert \underline{v} \rVert cos \theta
   \end{align*}

   Hence the angle $ \theta $ is  \[
     arccos \frac{ \underline{u} \cdot \underline{v}}{ \lVert \underline{u} \rVert \lVert \underline{v} \rVert }
   \] 


   For example, for vector $ \underline{v} = \begin{pmatrix} 2 \\ 6 \\0\\-1\\5 \end{pmatrix}$, find a vector $ \underline{u}$ where all the elements are non-zero, which is orthogonal to \underline{v} \\

   one such vector is \[
     \underline{u} = \begin{pmatrix} 
       -6 \\ 2 \\ 343738943948390 \\ 3 \\ 1  
     \end{pmatrix}
     :)
   \] 

\end{framed}

\subsection{The Cross Product}
\begin{framed}
   Note that unlike the dot product which works in \emph{all dimensions}, the cross product only works in
   \begin{itemize}
      \item $\mathbb{R}^3$
      \item $\mathbb{R}^2$, if you cheat :)
   \end{itemize}

   The dot product of two vectors $ \underline{u }$ and $ \underline{v}$ is 
   \begin{align*}
      \underline{u} \times \underline{v} = \begin{pmatrix} 
        u_2 v_3-u_3 v_2 \\  
        u_3 v_1-u_1 v_3 \\  
        u_1 v_2-u_2 v_1 
      \end{pmatrix}
   \end{align*}

   Note the anti-commutative property of the cross product 
   \begin{align*}
     \underline{u} \times \underline{v} = - \underline{v} \times \underline{u}
   \end{align*}

   \textbf{IMPORTANT - Commit to memory!! }

   Note the geometric understanding of the cross product, where the length of the cross product contains information about the length of  $ \underline{u}$ and $ \underline{v}$ and the angle between them $\theta$

    \begin{align*}
      \lVert \underline{u} \times \underline{v} \rVert = \lVert \underline{u} \rVert \lVert \underline{v} \rVert sin \theta
   \end{align*}

   The length of $ \underline{u} \times \underline{v}$ also gives the \emph{area of the parallelogram} with sides $ \underline{u}$ and $ \underline{v}$ \\

   Note that instead of detecting orthogonality like the dot product, the cross product detects parallel vectors \\

   Also note the \emph{mutual orthogonality} of the cross product $ \underline{u} \times \underline{v}$ to $ \underline{u}$ and $ \underline{v}$
\end{framed}

\subsection{The Scalar Triple Product}

\begin{framed}
   \textbf{Don't need to commit to memory!!} \\
   The scalar triple product takes as its arguments three vectors in $ \mathbb{R}^3$:
      $ \underline{u}$, 
      $ \underline{v}$, 
      $ \underline{w}$,
   And returns a scalar
   \[
     \underline{u} \cdot \left( \underline{v} \times \underline{w} \right)
   \] 

   However, know the geometric interpretation of the scalar triple product, which contains information about the 3-d volume spanned by $ \underline{u}, \underline{v}, \underline{w}$, in the shape of parallelopiped \\

   Also note the anti-symmetrical properties of the scalar triple product
  \begin{align*}
       &  \underline{u} \cdot \left( \underline{v} \times \underline{w} \right) \\
     = &  \underline{v} \cdot \left( \underline{w} \times \underline{u} \right) \\
     = &  \underline{w} \cdot \left( \underline{u} \times \underline{v} \right) \\
     = & - \underline{u} \cdot \left( \underline{w} \times \underline{v} \right) \\
     = & - \underline{w} \cdot \left( \underline{v} \times \underline{u} \right) \\
     = & - \underline{v} \cdot \left( \underline{u} \times \underline{w} \right) 
  \end{align*}
\end{framed}


\textbf{A note on PrepQuiz, Practice problems, and the Friday Quiz}
\begin{itemize}
   \item PrepQuiz is for conceptual understanding. Problems are different from those found in the Friday quiz
   \item Practice problems found on canvas are more similar to the problems expected on Friday
\end{itemize}


\subsection{Calculus with parameterized curves}

\begin{framed}
   For a curve 
   \begin{align*}
     \gamma (t) = \begin{pmatrix} 
       x(t) \\ y(t) \\ z(t)  
     \end{pmatrix}
   \end{align*}

   The derivative is trivial
   \begin{align*}
     \gamma ^{\prime} (t) = \begin{pmatrix} 
       x ^{\prime}(t) \\ y ^{\prime}(t) \\ z ^{\prime}(t)  
     \end{pmatrix}
   \end{align*}

   If the position vector is denoted by $ \gamma(t)$, then 
    \begin{align*}
       \text{velocity }& \underline{v} = \gamma ^{\prime} \\
       \text{speed }& s = \lVert \underline{v} \rVert  \\
       \text{acceleration }& \underline{a} = \gamma ^{\prime\prime}
   \end{align*}

   \textbf{Ability to decompose acceleration into its orthogonal components is NOT needed for 1410}
   
   The acceleration vector $ \underline{a}$ can be decomposed into
   \begin{align*}
      \underline{a} = \underline{a}_{TAN} + \underline{a}_{NORMAL}
   \end{align*}

   Note the unit tangent vector, especially note that all derivatives are w.r.t parameter $t$
   \begin{align*}
      \hat{\underline{T}} &= \frac{\gamma ^{\prime}}{ \lVert \gamma ^{\prime} \rVert } \\
   \end{align*}

   and the unit normal vector
   \begin{align*}
      \hat{\underline{N}} &= \frac{ \frac{d}{dt} \hat{ \underline{T}}}{ \lVert \frac{d}{dt} \hat{ \underline{T} }\rVert } \\
                          &= \frac{ \underline{T} ^{\prime}}{ \lVert \underline{T} \rVert }
   \end{align*}
\end{framed}


\subsection{Rules for derivatives of vectors}
\begin{framed}

   For two vectors $ \underline{ u}$ and $ \underline{v}$, 
   \begin{align*}
      \left( \underline{ u} \cdot \underline{v} \right) ^{\prime} &= \underline{u} ^{\prime} \cdot \underline{v} + \underline{u} \cdot \underline{v} ^{\prime} \\
      \left( \underline{ u} \times \underline{v} \right) ^{\prime} &= \underline{u} ^{\prime} \times \underline{v} + \underline{u} \times \underline{v} ^{\prime}
   \end{align*}
\end{framed}

\subsection{Arclength of a curve}
\begin{framed}
   For a curve \[
      \gamma: [a. b] \rightarrow \mathbb{R}^n
   \] 

   As covered in 1400, don't memorize complicated formulae for volume or work done! Write down \[
     l = \int dl
   \] where $dl$ can be understood as a velocity vector, hence the arclength element $dl$ is the "speed" of moving along curve $\times$ the "time elapsed" \[
     dl = \lVert \gamma ^{\prime} \rVert dt
   \] hence \[
   l = \int dl = \int_{a}^{b} \lVert \gamma ^{\prime} \rVert dt
   \] 


  
\end{framed}













