\chapter{Week 9 Lecture}

\section{Integration}

Integration was tricky in single variable calculus because of all the methods and tricks needed \\

In this course, we will only use integration by substitution in full generality. \\

This will be a slow week! Everyone is back at the same place! \\

\subsection{Integration in 1-D}

Recall the two kinds of integrals in 1-D \\

\begin{itemize}
   \item The definite integral
      \[
         \int_{x = a}^{b} f dx
      \] 

      The definite integral is a scalar value
   \item The indefinite integral
      \[
         \int_{}^{} f dx
      \] 

      The indefinite integral is a \textbf{class} of functions
\end{itemize}

The two objects are joined by the Fundamental Theorem of Integral Calculus 
\begin{itemize}
   \item From the indefinite integral, find the antiderivative, evaluate at the end points and take the difference
   \item You could think of the indefinite integral as a \textbf{means to an end}
\end{itemize}

In multivariable calculus, there is \textbf{no longer the indefinite integral}
\begin{itemize}
   \item The indefinite integral has no multivariate analogue
\end{itemize}

There are certain 1-D interpretations of the derivative and integral that do not persist into higher dimensions. For example
\begin{itemize}
   \item The derivative as \textit{the slope}
   \item The integral as \textit{the area}
\end{itemize}

Instead, we work with the intuition 
\begin{itemize}
   \item The derivative as a \textbf{linear transformation}
   \item The integral as a \textbf{mass} \\
\end{itemize}

\subsection{The integral as a mass}
Think of $f$ as the density as a function of space. The mass element $dM$ of a infinitesimal length $dx$ is given by
\[
  dM = f(x) dx
\] 

The full mass is 
\[
  M = \int dm
\] 

\begin{framed}
   The only formula needed for calculating integrals is 
   \[
     M = \int dm
   \] 

   \[
     L = \int dl
   \] 

   \[
     A = \int da
   \] 
\end{framed}

Now in the multivariate case \\

For a multivariate density function on domain D, of the form
\[
  f: \mathbb{R}^2 \rightarrow \mathbb{R}
\] 

Where the density at $x_i, y_i$ is given by $ f(x_i, y_i)$ \\

Consider the infinitesimal area element $dA$, its mass is given by
 \[
  dM = f dA
\] 
\[
  M = \int f dA
\] 

Consider that the dimensions of $dA $ is given by
\begin{itemize}
   \item height $dy$ 
   \item width $dx$
\end{itemize}

Hence we get
\[
  M = \int f dA = \iint f(x,y) dx dy
\] 

The double integral is really
\[
   \lim_{\substack{\triangle x \to 0 \\ \triangle y \to 0}} \sum_{j} \sum_{i} f(x_i, y_j) \triangle x_i \triangle y_j
\] 

Likewise in 3-D, for a function $f(x, y, z)$

\[
  M = \int f dV = \iiint f(x, y, z) dx dy dz
\] 

\textbf{General intuition about integrals}

If $f > 0$, what can you say about \[
   \iint_{D} f(x,y) dx dy
\] 

We can conclude that \[
   \iint_{D} f(x,y) dx dy > 0
\] 

\section{Computing integrals and the Fubini theorem}

\begin{framed}
   \textbf{Theorem}: Some complicated stuff

   \textbf{Idea}: For some function which depends on a bunch of variables
   \[
     f( \underline{x}) d \underline{x} 
   \]  
   where 
   \[
   \underline{x} = (x_1, x_2,  \hdots x_n)
   \] 
   \[
   d\underline{x} = dx_1, dx_2,  \hdots dx_n
   \] 

   The integral over $ \mathbb{R}^n$ with respect to the volume element $d \underline{x}$ is
   \[
      \int_{ \mathbb{R}^n} f ( \underline{x}) d \underline{x} =
        \int \left( 
           \hdots \left( 
              \int \left( 
                 \int f dx_1
              \right) dx_2
           \right) 
        \right)  dx_n
   \]  

   Most importantly: the order in which the integral is computed does not matter
  
\end{framed}

\textbf{Example}
For a function 
\[
  D = f(x, y) = x^2 + yy
\] over the domain (rectangle) bound by $x = 1, x = 3, y = 0, y = 1$ \\

The integral over $D $ is given by
\[
   \int_{D} x^2 + y\ dA = 
   \int_{y=0}^{1}\int_{x=1}^{3} x^2 + y\ dx dy = 
   \int_{x=1}^{3}\int_{y=0}^{1} x^2 + y\ dy dx
\]

Evaluating the integral
\[
\begin{align*}
   \int_{D} x^2 + y\ dA &= \int_{y=0}^{1}\int_{x=1}^{3} x^2 + y\ dx dy \\
                        &= \int_{y = 0}^{1} \left. \left(  
                           \frac{x^3}{3} + xy
                        \right) \right|_{x=1}^{3} dy \\
                        &= \int_{y=0}^{1} (9+3y) - ( \frac{1}{3} + y) dy \\
                        &=  \left. y^2 + \frac{26}{3}y \right|_{y=0}^{1} \\
                        &= \frac{29}{3}
\end{align*}
\] 

\textbf{Another example} \\

Consider a density function
\[
  f = x^2 + y^2
\] 


Compute the mass of the domain $D$ bound by \[
   y = x^{ \frac{1}{3}}, y = x^2
\] 

The set up of the double integral requires some thought \[
   M = \int_{y = 0}^{1} \int_{x=y^3}^{\sqrt{y}} x^2 + y^2 dx dy
\] 
Note that
\begin{itemize}
   \item on the outermost integral, there can be no variables in the boundaries of integration because the integral is a \textbf{scalar} value
   \item on the inner integrals, think about fixing $y$, and track the end points of the row at a given $y$, the end points should be a function $x = f(y)$. Computing the inner integral gives a density function of $y$
\end{itemize}

We can also set up the double integral as  \[
   M = \int_{x = 0}^{1} \int_{y = x^2}^{\sqrt{x}} x^2 + y^2 dy dx
\] 
Note that
\begin{itemize}
   \item This time, to write down the limits of the inner integral, we fix $x$. Then, we keep track of the \textbf{column} at a given $x$, and we read off the upper and lower bounds of that column. Computing the integral would give us a density function with respect to $x$
\end{itemize}

\section{A note not covered in lecture videos}

By Fubini's theorem, if all limits are constant, and if the integrand is a product of a function of one variable, then we can "split up the integrals"
\[
   \int_{c}^{d} \int_{a}^{b} f(x) g(y) dx dy = \int_{c}^{d} g(y) dy \int_{a}^{b} f(x) dx
\] 

An example
\[
   \int_{z = 2}^{5} \int_{y=-1}^{1} \int_{x=0}^{1} x^2 y^3 \sqrt{z} + x^4 z dx dy dz
\] 

This would take a while to compute, unless... \\

Note that the integral of $x^2 y^3 \sqrt{z}$ cancels out because $f(y) = y^3$ is an odd function, and the domain is symmetrical, hence $\int_{y = 1}^{1} \int \int x^2 y^3 \sqrt{z} dx dz dy = 0$ \\

Then, we can apply the above mentioned property, which states that the integral of a product is the product of integrals, provided all boundaries are constants \\


\begin{align*}
   \int_{z = 2}^{5} \int_{y=-1}^{1} \int_{x=0}^{1} x^2 y^3 \sqrt{z} + x^4 z dx dy dz &= \int_{z = 2}^{5} \int_{y=-1}^{1} \int_{x=0}^{1} dx dy dz \\
                                                                                     &= \int_{z=2}^{5}z dz \int_{y=-1}^{1} dy \int_{x=0}^{1} x^4 dx \\
                                                                                     &= \frac{21}{5}
\end{align*}


\section{A devilish example}

Prove the following result
\[
   \int_{x=-1}^{1} \int_{y=0}^{\sqrt{1-x^2}} \int_{z=0}^{\sqrt{1-x^2 - y^2}}  1+ x \ dz \dy\ dz = \frac{\pi}{3}
\] 

We start by looking at the domain \\

With respect to $z$ axis, $z$ is going from $z = 0$ to $z = \sqrt{1 - x^2 - y^2}$. Note that  $z^2 = 1 - x^2 - y^2$ is the equation for a unit sphere. \\

Think of the first integral of collapsing 3-d mass along the $z$ axis into the $x-y$ plane. \\

In the $xyz$ space, we are collapsing the upper hemisphere \\

In the $xy$ plane, for the $y$ integral, we are collapsing the upper semicircle \\

In the $x$ space, the domain goes from $-1 $ to $1$ \\

For the whole integral, our domain is the quarter sphere, where $z$ goes from 0 to 1, $y$ goes from 0 to 1, and $x$ goes from $-1$ to $1$

We split up the integral using the linear property of integrals

\[
   \int_{x=-1}^{1} \int_{y=0}^{\sqrt{1-x^2}} \int_{z=0}^{\sqrt{1-x^2 - y^2}}  1+ x \ dz \dy\ dz =
   \iiint \limits_{D}^{}  1 \ dz \dy\ dz + 
   \iiint \limits_{D}^{}  x \ dz \dy\ dz 
\] 

The first integral is the integral of the volume element
\[
   \int \limits_{x=-1}^{1} \int \limits_{y=0}^{\sqrt{1-x^2}} \int \limits_{z=0}^{\sqrt{1-x^2 - y^2}}  1 \ dz \dy\ dz  =  \iiint \limits_{D}^{} 1 dV = \frac{1}{4} \frac{4 \pi}{3}  = \frac{\pi}{3}
\] 

The second integral makes use of the fact that $f(x) = x$ is an odd function and the domain is symmetrical, hence it evaluates to 0

\section{Another example}

Recall that for $f$ on domain $D$ \\

The average is given by
\[
   \overline_{f} = \frac{
      \int_{D} f d \underline{x}
   }{
      \int_{D} 1 d \underline{x} 
   } = VOL(D)
\] 

\section{Another example}
Find the average of 
\[
  f = xy^2
\] 

Step 1: write the formula!!
\[
   \overline_{f} = \frac{\int_{D} f d \underline{x}}{\int_{D} 1 d \underline{x}}
\] 

Step 2: evaluate the denominator
\[
   \text{area of domain} = 2 \times 2 = 4
\] 

Step 3: evaluate the integral
\[
   \int_{0}^{2} \int_{0}^{2} xy^3\ dx\ dy = \left. \frac{x^2}{2} \right|_{0}^{2} \left. \frac{y^4}{4} \right|_{0}^{2} = 2 \times 4 = 8
\] 

Step 4
\[
  \frac{8}{4} = 2
\] 

\section{Another example with a weird domain}

what if the domain is the square from $(0,0)$ to $(2,2)$, excluding the square from $(0,0)$ to $(1,1)$ \\

step 1: compute the average on the domain $(0,0)$ to $(1,1)$
\[
   \overline{f} = \int_{0}^{1} \int_{0}^{1}  xy^3 dx dy = \frac{1}{2} \frac{1}{4} = \frac{1}{8}  
\] 

step 2: compute the average on the restricted domain
\[
   \overline{f}_{restricted} = \frac{2 \times 4 - \frac{1}{8}}{4 -1} = \frac{63}{24}
\] 





