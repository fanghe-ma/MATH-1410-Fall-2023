\chapter{Week 10 lectures}

This week, we will think about two applications of multivariate integration \\

\textbf{Physical mass}
The first intuition will require us to think of an integral as physical mass \\

The function $f$ is a function of density. To find the mass between $a$ and $b$, we integrate the mass element
\[
  M = \int dM
\] 
The mass element is given by
\[
  dM = f(x) dx
\] 
\[
   M = \int_{x = a}^{b} dx
\] 

\textbf{Probability}
The function $f$ is a probability density function on domain $D$
    \[
     f \sim PDF
    \] 

     where \[
      f(x) \geq 0, \int_D f = 1
     \] 


To find the probability, we integrate the probability element
\[
  P = \int dP
\] 

For $A \in D$, the probability that  $ \underline{x}$ is in $A $ is 
\[
  P( \underline{x}\ in\ A) = \int_A DP = \int_A f( \underline{x}) d \underline{x}
\] 

Normalized to 
\[
  P( \underline{x}\ in\ D) = 1
\] 

In $ \mathbb{R}^2$, the probability element in domain $D$ is\[
  dP = \frac{1}{AREA(D)} dx dy
\] 


Note that 
\[
   P(x = c) = \int_{\{c\}} dP = \int_{x=c}^{c} f(x) dx = 0
\] 

\section{Centroids \& Centers of mass}

For the centroid between $f$ and $g$ on domain $D$, whose area is $A$ \\

In high school the formula to memorize would have been
\[
   \overline{x} = \frac{1}{A} \int_{x=a}^{b} x (f(x) - g(x)) dx
\] 
\[
   \overline{y} = \frac{1}{2A} \int_{x=a}^{b} (f(x) - g(x))^2 dx
\] 

Instead, now with the language of double integrals, we can find $\overline{x}, \overline{y}$ using the average formula

 \[
    \overline{x} = \frac{1}{A} \iint_{D} x dA
\] 
 \[
    \overline{y} = \frac{1}{A} \iint_{D} y dA
\] 

\textbf{Centroid}
\[
   \overline{x} = \frac{\int_{D} x dV}{\int_{D} dV}
\] 

\textbf{However}, in calculating the centroid this way, we are assuming uniform density. If we want to find the center of mass, we need to use the \textbf{mass element}

\[
   \overline{x} = \frac{\int_{D} x dM}{\int_{D} dM}
\] 

where 
\[
  dM = f( \underline(x) d \underline{x}
\] 

\section{Expectation \& Spread}

For probability in $2D$, for a PDF of $f(x,y)$ over domain  $D$ such that $f(x,y) \geq 0$ and  $\iint_{D} f dxdy = 1$ \\

For two random variables $X, Y$ 
\[
   E(X) = \overline{x}  = \int_{D} x dP
\] 
\[
   E(Y) = \overline{y}  = \int_{D} y dP
\]  

The spread is measured by variance and standard deviation

\[
   V(X) = \int_{D} (x - E(X))^2 dP
\] 

\[
  \sigma(X) = \sqrt{V(X)}
\] 

Why doe the variance formula make sense? If we choose a good coordinate system (domain)  such as $E(X) = 0$, 
\[
   V = \int_{D} x^2 dP = \overline{x^2}
\] 

Now, it is easier to see why the variance is a measure of spread about the mean

\section{Moment of inertia}

For a domain $[-a, a]$ with uniform density, the moment of inertia is given by
\[
   \int_{-a}^{a} x^2 dM
\] 

In general, for a mass element $dM$ at a distance $r$ away from the axis, the integral element is the product of squared distance and the mass element
\[
  dI = r^2 dM
\] 

The moment of inertia is thus
\[
  I = \int dI = \int r^2 dM
\] 

Consider the symmetry between the two concepts \\
\begin{center}
   \begin{tabular}{|c | c|}
      \hline
      Physical & Probability \\ 
      \hline
      Variance & Moment of Inertia \\
      $V(X)$ & $I$ \\
      $\int x^2 dP$ & $\int r^2 dM$ \\
      \hline
      Standard Deviation & Radius of gyration \\
      $\sqrt{V(X)}$ & $R_g = \sqrt{ \frac{I}{M}}$ \\
      \hline
   \end{tabular}
\end{center}

\section{Things that we need to learn to compute this week}
In the language of physics
\begin{itemize}
   \item Mass
   \item Centroid
   \item Center of mass
   \item Moment of gyration
   \item Moments of inertia
   \item Radius of gyration
\end{itemize}

In the language of probability
\begin{itemize}
   \item Probability
   \item Expectation
   \item Variance
   \item Standard Deviation
\end{itemize}

\subsection{Example 1 - Probability}

Given a probability density function \[
   \rho = x^2 + y^2 
\] 

On a domain \[D = \{ \lVert x \rVert \leq 2, \lVert y \rVert \leq 3 \} \]

Note that total probability must be $1$
 \[
    \int \limits_{y = -3}^{3} \int \limits_{x = -2}^{2} C(x^2 + y^2) dx\ dy = 1
\] 

To find the constant $C$ such that the integral is 1
\begin{align*}
   \int \limits_{y = -3}^{3} \int \limits_{x = -2}^{2} C(x^2 + y^2) dx\ dy &= 
   \left. \left. C \left( \frac{x^3}{3}y + x \frac{y^3}{3} \right) \right|_{y = -3}^{3}  \right|_{x=-2}^{2} \\
                                                                           &= 104C
\end{align*}

Hence the probability density \[
  \rho = \frac{1}{104} \left( x^2 + y^2 \right) 
\] 

Now we can compute the probability, for example, to find
\begin{align*}
   P(\text{ $X \geq$ 1 }) &= \int_{\{ x \geq 1 \}} \rho\ dA  \\
                          & = \int_{y=-3}^{3} \int_{x=1}^{2} \frac{1}{104} \left( x^2 + y^2 \right) dx\ dy \\
                          &= \left. \left. \frac{1}{104} \left( \frac{x^3 y + xy^3}{3}\right)  \right|_{y=-3}^{3} \right|_{x=1}^{2}  \\
                          &= \frac{4}{13}
\end{align*}

Given the same probability density function, find $ P(\text{ $X \geq 1$ and $Y \leq -1$ }) $ \\

Given the same probability density function, find $ P(\text{ $X \geq 1$ or $Y \leq -1$ }) $

\subsection{Example - Moments of Inertia}

The moment of inertia is calculated by integrating the moment of inertia element
\[
   I = \int dI 
\]  Where $dI$ is given by
\[
   dI = r^2 dM
\] 

Where
 \begin{itemize}
    \item $r$ is the radius to axis of rotation
    \item $dM$ is the mass element, equals to $\rho dV$
\end{itemize}

\subsection{Example - More probability}
For a given PDF \[
   \rho = C e^{- \alpha x}
\] 

on a domain
\[
   D = [0, \infty)
\] 

Integrating the PDF from $0$ to infinity and setting total probability to 0, we solve for $C = \alpha$, hence the PDF is
\[
   \rho = \alpha e^{- \alpha x}
\] 

We can define a joint PDF for arbitrary number of variables by taking the product of $\rho_i$
\[
   \rho(x_1, x_2 \hdots x_n) = \alpha_1 \alpha_2 \hdots \alpha_n e^{- \alpha_1 x - \alpha_2 x - \hdots \alpha_n x}
\] 

